
%--------------------------------------------------------------------------
\chapter{LITERATURE REVIEW}

%--------------------------------------------------------------------------
\section{Computers in the Classroom}

One of the most notable aspects of computer-based instruction and learning is its ability to replace devices which have been associated to the classroom for over a century, namely the blackboard, textbooks, dictionaries, notebooks, pencils, erasers, and pens. More subtly, however, the computer is capable of replacing rulers and other sketching devices with an immense increase in speed and accuracy, which has a fundamental effect on influencing the development of intuition around a yet unknown topic. If, on one hand, teaching devices in use today have not evolve much since the nineteen hundreds, on the other, the demands for knowledge in our society are rocketing, making the adoption of personal computers as teaching and learning tools all that more crucial \cite[77]{Roschelle2000}. One way of evaluating just how more effective computer-centered instruction can be is based on cognitive theories of learning, which suggest that the presence of active engagement, participation in groups, frequent interaction and feedback, as well as connections to real-world contexts are key to achieving said efficacy. These tenets go hand-in-hand with the characteristics, resources, and physical structure of the classroom, indicating that a careful application of computer technology to aid learning can in fact better suit the way individuals learn than a traditional setting \cite[79]{Roschelle2000}. We discuss in detail each of these four premises below.

We refer to the traditional classroom practice of teaching through lectures supported by textbooks as the \emph{transmission} model of learning, which is characterized by putting the student in a very passive role. As a consequence, the transmission model does not engage the student in applying recently learned materials to situations beyond the context of classroom and textbook exercises. I some sense, the motivation for learning itself becomes clouded by the very model. This practice may contradict the idea of actively building knowledge from interactions, hands-on experience, and reasoning about the subject at hand. Computers can very effectively provide hands-on experience and can help build knowledge through data interpretation in the form of plots, graphs, videos, sounds, and internet searches, to name but a few. What is yet more remarkable is the agility with which all occurs. In cases where the learning style of a particular individual antagonizes the traditional transmission model, computers can further be adapted to suit more particular needs, such as a combined approach \cite[80]{Roschelle2000}.

%%--------------------------------------------------------------------------
%\subsection{participation in groups}
%
%``Social contexts give students the opportunity to successfully carry out more complex skills than they could execute alone. Performing a task with others provides an opportunity not only to imitate what others are doing, but also to discuss the task and make thinking visible. Much learning is about the meaning and correct use of ideas, symbols, and repre- sentations. Through informal social conver- sation and gestures, students and teachers can provide explicit advice, resolve misun- derstandings, and ensure mistakes are cor- rected. In addition, social needs often drive a child’s reason for learning. Because a child’s social identity is enhanced by partici- pating in a community or by becoming a member of a group,25 involving students in a social intellectual activity can be a powerful motivator and can lead to better learning than relying on individual desk work. ... Some of the most prominent uses of computers today are communications oriented, and networking technologies such as the Internet and digital video permit a broad new range of collaborative activities in schools.'' \cite[80]{Roschelle2000}
%
%``Reports from researchers and teachers suggest that students who par- ticipate in computer-connected learning net- works show increased motivation, a deeper understanding of concepts, and an increased willingness to tackle difficult questions.'' \cite[81]{Roschelle2000}
%
%%--------------------------------------------------------------------------
%\subsection{frequent interaction and feedback}
%
%``In traditional classrooms, students typically have very little time to interact with materi- als, each other, or the teacher.36 Moreover, students often must wait days or weeks after handing in classroom work before receiving feedback. In contrast, research suggests that learning proceeds most rapidly when learn- ers have frequent opportunities to apply the ideas they are learning and when feedback on the success or failure of an idea comes almost immediately.'' \cite[81]{Roschelle2000}
%
%``Unlike other media, computer technol- ogy supports this learning principle in at least three ways. First, computer tools them- selves can encourage rapid interaction and feedback. For example, using interactive graphing, a student may explore the behav- ior of a mathematical model very rapidly, getting a quicker feel for the range of varia- tion in the model. If the same student graphed each parameter setting for the model by hand, it would take much longer to explore the range of variation. Second, computer tools can engage students for extended periods on their own or in small groups; this can create more time for the teacher to give individual feedback to partic- ular children.38 Third, in some situations, computer tools can be used to analyze each child’s performance and provide more timely and targeted feedback than the stu- dent typically receives.'' \cite[81]{Roschelle2000}
%
%``The most sophisticated applications of computers in this area have tried to trace stu- dents’ reasoning process step by step, and provide tutoring whenever students stray from correct reasoning.'' \cite[82]{Roschelle2000}
%
%%--------------------------------------------------------------------------
%\subsection{connections to real-world contexts}
%
%``A vast literature on this topic sug- gests that, to develop the ability to transfer knowledge from the classroom to the real world, learners must master underlying con- cepts, not simply memorize facts and solu- tion techniques in simplified or artificial contexts.14 But typical problem-solving assignments do not afford students the opportunity to learn when to apply particu- lar ideas because it is usually obvious that the right ideas to apply are those from the immediately preceding text. ... Computer technology can provide stu- dents with an excellent tool for applying concepts in a variety of contexts, thereby breaking the artificial isolation of school sub- ject matter from real-world situations.'' \cite[82]{Roschelle2000}
%
%``Through the Internet, students from around the world can work as partners to scientists, business- people, and policymakers who are making valuable contributions to society.'' \cite[83]{Roschelle2000}
%
%%--------------------------------------------------------------------------
%\subsection{expanding what children learn}
%
%``In addition to supporting how children learn, computer-based technology can also improve what children learn by providing exposure to ideas and experiences that would be inaccessible for most children any other way. For example, because synthesiz- ers can make music, students can experi- ment with composing music even before they can play an instrument.'' \cite[84]{Roschelle2000}
%
%``The most interesting research on the ways technology can improve what children learn, however, focuses on applications that can help students understand core con- cepts in subjects like science, math, and lit- eracy by representing subject matter in less complicated ways. Research has demon- strated that technology can lead to pro- found changes in what children learn. By using the computers’ capacity for simula- tion, dynamically linked notations, and interactivity, ordinary students can achieve extraordinary command of sophisticated concepts. ... For example, technology using dynamic diagrams—that is, pictures that can move in response to a range of input—can help students visual- ize and understand the forces underlying various phenomena.'' \cite[86]{Roschelle2000}
%
%%--------------------------------------------------------------------------
%\subsection{Dynamic, Linked Notations}
%
%``Students can explore changes rapidly in the notation by dragging with a mouse, as opposed to slowly and painstakingly rewrit- ing the changes. Students can see the effects of changing one notation on another, such as modifying the value of a parameter of an equation and seeing how the resulting graph changes its shape. Students can easily relate mathematical symbols either to data from the real world or to simulations of familiar phenomena, giving the mathematics a greater sense of meaning. Students can receive feedback when they create a notation that is incorrect. (For example, unlike with paper and pencil, a computer can beep if a student tries to sketch a nonsensical mathematical function in a graph, such as one that “loops back” to define two different y values for the same x value.)'' \cite[88]{Roschelle2000}
%
%%--------------------------------------------------------------------------
%\subsection{Social Studies, Language, and the Arts}
%
%``Similar software can provide interactive media environments for classes in the arts. An emergent theme in many computer- based humanities applications is using tech- nology that allows students to engage in an element of design, complementing and enhancing the traditional emphasis on appreciation.'' \cite[88]{Roschelle2000}
%
%``Based on the research to date, the strongest evidence showing positive gains in learning tends to focus on applications in science and mathematics for upper elementary, middle, and high school students.'' \cite[78]{Roschelle2000}
%
%``Although there are fewer studies on the effectiveness of technology use in these other subject areas...'' \cite[89]{Roschelle2000}
%
%``In one innovative project, elementary and middle school children alternate between playing musical instruments, singing, and programming music on the computer using Tuneblocks, a musical ver- sion of the Logo programming language.69 Compelling case studies show how using this software enables ordinary children to learn abstract musical concepts like phrase, figure, and meter—concepts normally taught in college music theory classes.'' \cite[90]{Roschelle2000}
%
%%--------------------------------------------------------------------------
%\subsection{Challenges}
%
%``Studies show that a teacher’s ability to help students depends on a mastery of the struc- ture of the knowledge in the domain to be taught.73 Teaching with technology is no dif- ferent in this regard.'' \cite[90]{Roschelle2000}
%
%``By networking with mentors and other teachers electronically, teachers can overcome the isolation of the classroom, share insights and resources, support one another’s efforts, and engage in collabora- tive projects with similarly motivated teach- ers. ... Teachers who succeed in using technol- ogy often make substantial changes in their teaching style and in the curriculum they use.'' \cite[91]{Roschelle2000}
%
%``One of the biggest barriers to introducing effective technology applications in class- rooms is the heavy focus on student perfor- mance on district- or state-mandated assessments and the mismatch between the content of those assessments and the kinds of higher-order learning supported most effectively by technology.79 This mismatch leads to less time available for higher-order instruction and less appreciation of the impact technology can have on learning. Moreover, it will be dif- ficult, if not impossible, to demonstrate the contribution of technologies in developing students’ abilities to reason and understand concepts in depth without new kinds of assessments.'' \cite[91]{Roschelle2000}
%
%``These studies suggest that the relationship between tech- nology use and education reform is recipro- cal: although technology use helps support school change, school change efforts also help support effective use of technology.'' \cite[92]{Roschelle2000}
