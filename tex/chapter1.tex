
\chapter{INTRODUCTION}

Formal languages lend precision and flexibility to music specification because they require that musical ideas be turned into abstract symbols and stipulated explicitly.

Herein lies both the advantage and disadvantage of linguistic interaction with a computer music system.

The advantage is that formalized and explicit instructions can yield a high degree of control.

To create an imagined effect, composers need only specify it precisely.

They can easily stipulate music that would be difficult or impossible to perform by human beings.

In some cases, a linguistic specification is much more efficient than gestural input would be.

This is the case when a single command applies to a massive group of events, or when a short list of commands replaces dozens of pointing and selecting gestures.

The shell scripts of Unix operating systems are a typical example of command lists (Thompson and Ritchie 1974).

These advantages turn into a disadvantage when simple things must be coded in the same detail and with the same syntactic overhead as complicated things.

For example, with an alphanumeric language, envelope shapes that could be drawn on a screen in two seconds must be plotted out on paper by hand and transcribed into a list of numerical data to be typed by the composer.

For many tasks, graphical editors and visual programming systems, in which the user selects and interconnects graphical objects, are more effective and easier to use than their textual counterparts (see chapter 16).

Some languages are interactive; one can type individual statements and each of them is interpreted in turn.

This can occur in a concert situation, but the slow information rate of typing — not to mention the mundane stage presence of a typist — precludes this approach in fast­paced real­time music­making.

Gestural control through a musical input device is more efficient and natural.

Hence, languages for music, although important, do not answer all musical needs.

In the ideal, music languages should be available alongside other kinds of musical interaction tools.

\cite[785-786]{Roads1995}.