
\chapter{LITERATURE REVIEW}

% TODO: see Howe 1975

Arguably the first notable attempt to design a programming language with an explicit intent of processing sounds and making music was that of \emph{Music I}, created in 1957 by Max Mathews. The language was indented to run on an IBM 704 computer, located at the IBM headquarters in New York City. The programs created there were recorded on digital magnetic tape, then converted to analog at Bell Labs, where Mathews spent most of his career as an electrical engineer. \emph{Music I} was capable of generating a single waveform, namely a triangle, as well as assigning duration, pitch, amplitude, and the same value for decay and release time. \emph{Music II} followed a year later, taking advantage of the much more efficient IBM 7094 to produce up to four independent voices chosen from 16 waveforms. With \emph{Music III}, Mathews introduced in 1960 the concept of a \emph{unit generator}, which consisted of small building blocks of software that allowed composers to make use of the language with a lot less effort and required background. In 1963, \emph{Music IV} introduced the use of macros, which had just been invented, although the programming was still done in assembly language, hence all implementations of the program remained machine-dependent. With the increasing popularity of Fortran, Mathews designed \emph{Music V} with the intent of making it machine-independent, at least in part, since the unit generators' inner loops were still programmed in machine language. The reason for that is the burden these loops imposed on the computer \cite[15-17]{Roads1980}.

\section{Software Synthesis Languages}

Since Mathews' early work, much progress has been made, and a myriad of new programming languages that support sound processing, as well as domain-specific languages whose sole purpose is to process sounds or musical events, have surfaced. In \cite{Roads1995}, we see an attempt to classify these languages according to the specific aspect of sound processing they perform best. The first broad category described is that of \emph{software synthesis languages}, which compute samples in non-real-time, and are implemented by use of a text editor with a general purpose computer. The \emph{Music N} family of languages consist of software synthesis languages, and Tab.~\ref{tab:synthesis} presents a list of software synthesis languages developed until 1991. A characteristic common to all software synthesis languages is that if a toolkit approach to sound synthesis, whereby using the toolkit is straightforward, however customizing it to fulfill particular needs often require knowledge of the programming language in which the toolkit was implemented. This approach provides great flexibility, but at the expense of a much steeper learning curve. Another aspect of software synthesis languages is that they can support an arbitrary number of voices, and the time complexity of the algorithms used only influences the processing time, not the ability to process sound at all, as we see with real-time implementations. As a result of being non-real-time, software synthesis languages usually lack controls that are gestural in nature. Yet, software synthesis languages are capable of processing sounds with a very fine numerical detail, although this usually translates to more detailed, hence verbose code. Software synthesis languages, or non-real-time features of a more general-purpose language, are sometimes required to realize specific musical ideas and sound-processing applications that are impossible to realize in real time \cite[783-787]{Roads1995}.

Within the category of software synthesis languages, we can further classify those that are \emph{unit generator languages}. This is exactly the paradigm originally introduced by \emph{Music III}. In them, we usually have a separation between an orchestra section, and a score section, often given by different files and sub-languages. A unit generator is more often than not a built-in feature of the language. Unit generators can generate or transform buffers of audio data, as well as deal with how the language interacts with the hardware, that is, provide sound input, output, or print statements to the console. Even though one can usually define unit generators in terms of the language itself, the common practice is to define them as part of the language implementation itself. Another characteristic of unit generators is that they are designed to take as input arguments the outputs of other unit generators, thus creating a signal flow. This is implemented by keeping data arrays in memory which are shared by more than one UG procedure by reference. The score sub-language usually consists of a series of statements that call the routines defined by the orchestra sub-language in sequential order, often without making use of control statements. Another important aspect of the score sub-language is that it defines function lookup tables, which are mainly used to generate waveforms and envelopes. When \emph{Music N} languages became machine-independent, function generating routines remained machine-specific for a period of time, due to performance concerns. On the other hand, the orchestra sub-language is where the signal processing routines are defined. These routines are usually called instruments, and basically consist of new scopes of code where built-in functions are dove-tailed, ultimately to a unit generator that outputs sound or a sound file \cite[787-794]{Roads1995}.

The compilation process in \emph{Music N} languages consists usually of three passes. The first pass is a preprocessor, which optimizes the score that will be fed into the subsequent passes. The second pass simply sorts all function and instrument statements into chronological order. The third pass then executes each statement in order, either by filling up tables, or by calling the instrument routines defined in the orchestra. The third pass used to be the performance bottleneck in these language implementations, and during the transition between assembly and Fortran implementations, these were the parts that remained machine-specific. Initially, the output of the third pass consisted of a sound file, but eventually this part of the compilation process was adapted to generate real-time output. At that point, defining specific times for computing function tables became somewhat irrelevant.

In some software synthesis languages, the compiler offers hooks in the first two passes so that users can define their own sound-processing subroutines. In any cases, these extensions to the language were given in an altogether different language. With \emph{Common Lisp Music}, for example, one could define the data structures and control flow in terms of Lisp itself, whereas \emph{MUS10} supported the same features by accepting Algol code. In \emph{Csound}, one can still define control statements in the score using Python. Until \emph{Music IV} and its derivatives, compilation was sample-oriented. As an optimization, \emph{Music V} introduced the idea of computing samples in blocks, where audio samples maintained their time resolution, but control statements could be computed only once per block. Of course, if the block size is one, than we compute control values for each sample, as in the sample-oriented paradigm. Instead of defining a block size, however, one defines a control rate, which is simply the sampling rate times the reciprocal of the block size. Hence a control rate that equals the sampling rate would indicate a block size of one. With \emph{Cmusic}, for instance, we specify the block size directly, a notion that is consistent with the current practice of specifying a vector size in real-time implementations. The idea of determining events in the language that could be computed at different rates required some sort of type declaration. In \emph{Csound}, these are given by naming conventions: variables whose names start with the character `a' are audio-rate variables, `k' means control rate, and `i'-variables values are computed only once per statement. \emph{Csound} also utilizes naming conventions to determine scopes, with the character `g' indicating whether a variable is global \cite[799-802]{Roads1995}.

\section{Real­-Time Synthesis Control Languages}

Some of the very first notable attempts to control the real-time synthesis hardware were made at the \emph{Institut de Recherche et Coordination Acoustique/Musique} in the late seventies. Many of these early attempts made use of programming languages to drive the sound synthesis being carried out by a dedicated DSP. Tab.~\ref{tab:realtime} presents a list of real-time synthesis control languages developed until 1991. At first, most implementations relied on the concept of a \emph{fixed-function} hardware, which required significantly simpler software implementations, as the latter served mostly to control a circuit that had an immutable design and function. An example of such fixed-function implementations would be an early frequency-modulation synthesized, which contained a dedicated DSP for FM-synthesis, and whose software implementation would only go as far as controlling the parameters thereof. Often, the software would control a chain of interconnected dedicated DSP's, which would in turn produce envelopes, filters, and oscillators. The idea of controlling parameters through software, while delegating all signal processing to hardware, soon expanded beyond the control of synthesis parameters, and into the sequencing of musical events, like in the New England Digital Synclavier. Gradually, these commercial products began to offer the possibility of changing how exactly this components were interconnected, what is called a \emph{variable-function} DSP hardware. Interconnecting these components through software became commonly called \emph{patching}, as an analogy to analog synthesizers. The idea of patching brought more flexibility, but imposed a steeper learning curve to musicians. Eventually, these dedicated DSP's were substituted by general-purpose computers, wherein the entire chain of signal processing would be accomplished via software \cite[802-804]{Roads1995}.

Commonly in a fixed-function implementation there is some sort of front panel with a small LCD, along with buttons and knobs to manage user input. In the case of a keyboard instrument, there is naturally a keyboard to manage this interaction, as well. The purpose of the embedded software is then to communicate user input to an embedded system which contains a microprocessor and does the actual audio signal processing, memory management, and audio input/output. All software is installed in some read-only memory, including the operating system. With the creation of the \emph{Musical Instrument Digital Interface} standard in 1983, which was promptly absorbed my most commercial brands, the issue of controlling sound synthesis hardware transcended the interaction with keys, buttons, and sliders, and became a matter of software programming, as one could easily communicate with dedicated hardware, by means of a serial interface, MIDI messages containing discrete note data, continuous controller messages, discrete program change messages, as well as system­-exclusive messages. As a trend, many MIDI libraries were written at the time for general-purpose programming languages such as APL, Basic, C, Pascal, Hypertalk, Forth, and Lisp. In addition, most descendants of the \emph{Music N} family of languages began to also support MIDI messages as a way to control dedicated hardware \cite[804-805]{Roads1995}.

The implementation of a software application to control variable-function DSP hardware is no mundane task, as it requires knowledge of digital signal processing, in addition to programming in a relatively low level language. Dealing with issues of performance, memory management, let alone the mathematics required to process buffers of audio samples, often imposes an unsurmountable burden to musicians. Many solutions were invented in order to work around this difficulties, including the use of graphic elements and controllers, but ultimately it was the concept of a unit generator, borrowed from software synthesis languages, that most influenced the creation of higher-level abstractions that were more suitable for musicians. This is notably the case of the \emph{4CED} language, which was developed at IRCAM in 1980, and owed greatly to \emph{Music IV} and \emph{Music V}. The resemblance extended as far as to comprise a separate orchestra sub-language for patching unit generators, a score sub-language, and a third command sub-language for controlling effects in real-time, as well as to link both orchestra and score to external input devices such as buttons and potentiometers. The hardware these languages drove was IRCAM's 4C synthesizer. The result of nearly a decade of research at IRCAM culminated in \emph{Max}, a visual programming language that remains to this day one of the most important real-time tools for musicians. \emph{Max}, which will later be discussed in more detail, eventually transcended its hardware DSP and implemented itself in C the sound-generating routines. But that was not until the 2000's, ten years after it became a commercial software application, independent of IRCAM \cite[805-806]{Roads1995}.

\begin{example}
	\cite[809]{Roads1995}
	\emph{Music 1000} is a descendant of the \emph{Music N} family of languages that was designed to drive the Digital Music Systems DMX­1000 signal processing computer, in which we can clearly observe the unit-generator concept in action:
	\begin{lstlisting}
	fnctn func1, 512, fourier, normal, 1, 1000
	instr 1
		kscale amp, knob1, 0, 10000
		kscale freq, knob2, 20, 2000
		oscil x8, #func1, amp, freq
		out x8
	endin
	\end{lstlisting}
	In the code above, a \emph{fnctn} statement assigns to variable \emph{func1} an array of 512 samples using a \emph{fourier} series of exactly one harmonically-related sine, whose (trivial) sum is \emph{normal}-ized. The amplitude of 1000 is then meaningless, but a required argument. In fact, \emph{func1} takes a variable number of arguments, where for each harmonic partial, the user specifies a relative amplitude. The block that follows defines an instrument, in which the unit generator \emph{oscil} takes as arguments the output of three other unit generators, which are respectively the wavetable previously computed, as well as amplitude and frequency parameters, whose values are in turn captured by two knobs attached to the machine. The knobs produce values between 0 and 1, and the subsequent arguments to \emph{kscale} are scaling parameters. Finally, \emph{out} is a unit generator that connects the output of \emph{oscil} to the digital-to-analog converter.
\end{example}

\section{Music Composition Languages}

Between the 1960's and the 1990's, many programming languages were devised to aid music composition. As a noticeable trend, one can define two categories among those languages, namely those that are \emph{score input languages}, and those that are \emph{procedural languages}. The main difference between the two categories is that, in the former, some representation of a musical composition is already at hand, hence score input languages provide a way to encode that information. This could be a score, a MIDI note list, or even some graphical representation of music. In the latter category, the language provides, or helps define procedures that are used to generate musical material, a practice that is often called \emph{algorithmic} music composition. One outstanding characteristic of score input languages is how verbose and complex they can become, depending on the musical material they are trying to represent. This difficulty influenced the devising of many alternatives to textual programming languages, such as the use of scanners in the late 1990's by Neuratron's \emph{PhotoScore}, an implementation which was predicted by composer Milton Babbitt as early as in 1965. Before the advent of MIDI, however, programming languages were indeed the user interface technology of choice, or lack thereof, to design applications meant for analyzing, synthesizing, and printing musical scores. Tab.~\ref{tab:score} presents a list of score input languages developed until 1991. With the widespread adoption of the MIDI standard in the mid-1980's, whereby one can input note events by performing on a MIDI instrument, combined with the advancements in graphical user interfaces of the mid-1990's, the creation and maintenance of score input languages has faced a huge decline. What is even worse, the paradigm of a musical score is itself inadequate for computer music synthesis, in that a score is more often than not a very incomplete representation of a musical piece, often omitting a great deal of information. It is the job of a musical performer to provide that missing information. In this sense, \emph{procedural languages} are much better suited for computer performance, but that comes at the cost of replacing the score paradigm altogether \cite[811-813]{Roads1995}.

In 2018, a few \emph{score input languages} remain, despite the vast predominance of graphical user interfaces as a means to input notes to a score. \emph{MusiXTex} is a surviving example that compiles to \latex, which in turn compiles to PDF documents. It was created in 1991 by Daniel Taupin. The language has such unwieldy syntax, that often a preprocessor is required for more complex scores. One famous such processors is \emph{PMX}, a \emph{FORTRAN} tool written by Don Simons in the late 1990's. Another was \emph{MPP}, which stands for MusiXTex Preprocessor, created by Han-Wen Nienhuys and Jan Nieuwenhuizen in 1996, and which eventually became \emph{LilyPond}, arguably the most complete surviving score input language today. \emph{LilyPond} has a much simpler syntax than that of \emph{MusiXTex}, however not nearly as simple as \emph{ABC} music notation, a language that much resembles \emph{Musica} and which is traditionally used in music education contexts. A package written by Guido Gonzato is available in \latex which can produce simple scores in \emph{ABC} notation. Its simplicity comes, however, at the expense of incompleteness. Finally, it is worthwhile to mention a music-notation specific standard that has emerged in the mid-2000's, namely the \emph{MusicXML} standard. Heavily influenced by the industry, it was initially meant as an object model to translate scores between commercial applications where the score input method was primarily graphical, and whose underlying implementation was naturally object-oriented. \emph{MusicXML} is extremely verbose, and borderline human-readable. It is, however, very complete, to the point of dictating what features an object-oriented implementation should comprise in order to be aligned with the industry standards. In recent years, many rumors have surfaced to make \emph{MusicXML} an Internet standard, such as that of \emph{Scalable Vector Graphics}, however nothing concrete has been established.

\begin{example} \label{ex:musixtex}
	\cite[812]{Roads1995}
	\emph{Musica} was developed at the Centro di Sonologia Computazionale in Padua, Italy, and is particularly interesting in its interpreter compiles programs into \emph{Music V} note statements. 
	\begin{lstlisting}
	4'AGAG / 4.A8G2E / 4DDFD / 2ED
	\end{lstlisting}
	In the code above, all numbers indicate note duration, that is, 4 is a quarter-note, 8 is an eighth-note, and 2 is half-note, with dots indicating dotted durations. The letters indicate pitch, and the apostrophe indicates octave such that '$A = 440$Hz, and ''$A = 880$Hz. Finally, the slash indicates a measure. The code below, on the other hand, shows an example of the very same musical material expressed in \emph{MusiXTex}. One can immediately notice the difference in implementation by the sheer amount of code required to express basically the same symbols.
%	\begin{lstlisting}[language=[LaTeX]{TeX}]
%	\begin{music}
%		\generalmeter{\meterfrac44}
%		\startextract
%		\Notes \qu{h g h g} \en \bar
%		\Notes \qup{h} \cu{g} \hu{e} \en \bar
%		\Notes \qu{d d f d} \en \bar
%		\Notes \hu{e d} \en
%		\endextract
%	\end{music}
%	\end{lstlisting}
	In the snippet above, many commands, despite verbose, are quite self-explanatory. Some others, however, are not. The \textbackslash{qu} command means a quarter-note with a stem pointing upward, whereas the \textbackslash{Notes} command actually means how notes should be spaced. The more capital letters, the more spacing between the notes, that is, \textbackslash{NOTes} is more spaced out than \textbackslash{NOtes}. Finally, in addition to supporting the same apostrophes as \emph{Musica} for defining octave, \emph{MusiXTex} also supports other letters, as well as capitalizations thereof. In the example above, we have $h = 440$Hz, whereas $a = 220$Hz. Below is the output of the \emph{MusiXTex} code:
	\begin{music}
		\generalmeter{\meterfrac44}
			\startextract
			\Notes \qu{h g h g} \en \bar
			\Notes \qup{h} \cu{g} \hu{e} \en \bar
			\Notes \qu{d d f d} \en \bar
			\Notes \hu{e d} \en
		\endextract
	\end{music}
\end{example}

% 813: Procedural Composition Languages

Procedural composition languages go beyond the representation of traditional scores to support the unique possibilities of computer music.

These possibilities include alternative tunings, multiple envelopes for control of timbres and spatial paths, voice interplay, performer interaction, and compositional procedures.

These languages let composers specify music algorithmically.

They represent the flow of music as a collection of interacting processes.

Two advantages of representing a compositional process as a program stand out.

First, the compositional logic is made explicit, creating a system with a degree of formal consistency.

Abstract formal unity in a composition is prized by many music theorists and composers.

Second, rather than abdicating decision-­making to the computer, composers can use procedures to extend control over many more processes than they could manage with manual techniques.

Here are a few examples of how algorithms can expand the scope of compositional decisions:

1. Controlling the micro-frequency variations among the partials of a given tone

2. Sifting through massive amounts of data to select a specified sound or sound combination

3. Sending sounds on precise spatial paths computed according to composer­specified rules

4. Generating numerous variations quickly to provide the composer with many alternatives of a given sound or series of sounds

5. Realizing complex polyphonic textures that would otherwise be impossible to perform

6. Generating algorithmic music accompaniment in real time based on a performer's input

No "standard" language for procedural composition dominates the field.

This is not surprising, since there are so many diverse approaches to composition.

Table 17.4 lists examples of procedural composition languages.

Much of this software is organized as a collection of routines that are called from a traditional programming language, or are embedded within a special musical dialect of that language.

The boundary is blurry between some of the procedural composition languages cited here and the composition "micro-worlds," "toolkits,'' "environments," and "libraries" mentioned in chapter 18.

% 814: Music Languages Embedded in Programming Languages

Extensible languages support the notion of a specialized embedded language or micro-world written in terms of the original language.

The advantage of this implementation is that the facilities of the general language are always available to the sub-language.

A music composition language, for example, can always count on the general language for routines to handle floating­point arithmetic, files, input/output, window management, and other utilitarian tasks.

The power of embedding a composition language in a general­purpose programming language is also reflected in languages like MPL (Nelson 1977, 1980) embedded in APL, and Pla (Schottstaedt 1983, 1989a) em­bedded in Sail (Pla version 1) and Lisp (Pla version 2).

An intrinsically open language like Lisp encourages embedded languages.

For example, the composition languages MIDI­LISP (Boynton et al. 1986), FORMES (Rodet and Cointe 1984), Esquisse (Baisnée et al. 1988), PatchWork (Laurson and Duthen 1989; Barrière, Iovino, and Laurson 1991), Canon (Dannenberg 1989a), Flavors Band (Fry 1984), Symbolic Composer (Tonality Systems 1993), Lisp Kernel (Rahn 1990), and Common Music (Taube 1991) are all embedded in Lisp.

A number of software synthesis languages can be instructed to call user­written compositional procedures coded in general programming languages.

Examples include Music V, through user­written Fortran PLF subroutines (Mathews 1969), the Cscore part of Cmusic (Moore 1982; 1990), and the Music 4C language (Beauchamp and Aurenz 1985; Gerrard 1989; Beauchamp 1992b).

Since Cscore and Music 4C are embedded in the C programming language, composers are free to use the resources of that language and its operating system environment in specifying a composition.

The Music Kit (Jaffe 1989) can be used as a composition language for those versed in the Objective-C programming language.

% 817: Conclusion

Textual languages are a precise and flexible means of controlling a computer music system.

In synthesis, score entry, and composition, they can liberate the potential of a music system from the closed world of canned software and preset hardware.

In the mid­-1980's it looked as if the Music N synthesis language dynasty might languish, due to the spread of inexpensive synthesizers.

But synthesizers have never quite matched the flexibility of the Music N model.

Modern implementations of Music N—embedded in powerful programming languages and running on fast hardware—have infused new life into this most traditional approach to computer music.

The note list of the Music N languages exists in dozens of forms.

In MIDI systems, however, the use of a score language is less common, since most music can be entered by other means (such as a music keyboard, notation program, or scanner).

For musicological applications that involve score analysis, however, a text­-based score representation may be optimum.

New textual languages for procedural composition continue to be developed, but there is a strong parallel trend toward interactive programs with graphical interfaces.

Instead of typing text, one patches together icons, draws envelopes, and fills in templates.

The textual language representation supporting the graphics is hidden from the user.

Early programming languages for music tended to favor machine efficiency over ease of use.

The present trend in programming has shifted from squeezing the last drop of efficiency out of a system to helping the user manage the complexity of layer upon layer of software and hardware.

The most common solution to this problem is object­-oriented programming (see chapter 2), and compositional applications are no exception to this trend (Pope 1991b).
