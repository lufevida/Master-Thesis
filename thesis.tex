%&latex
% UF Sample ETD Main Document Fall 2014

%\documentclass[12pt,final,CPage]{ufthesis} % Use this line for Windows OS
\documentclass[12pt,dvipdfmx,final,CPage]{ufthesis} % Use this line for Macintosh/Linux

% Macintosh and Linux users - If you get a dvipdfm file not found error change dvipdfm to dvipdfmx here and in the packages.tex file graphicx and hyperref packages


\usepackage{graphicx}
\usepackage{amsmath}
\usepackage{amsthm}
\usepackage{tabularx}
\usepackage{url}
\usepackage[letterpaper,hmargin=1in,vmargin=1in]{geometry}
\usepackage{lscape}
\usepackage{hanging}
\usepackage{longtable}
\usepackage{amsfonts}
\usepackage{amssymb}
\usepackage[cmbright]{sfmath}
\usepackage{subfigure}
\usepackage{rotating}
\usepackage{calc}
\usepackage{setspace}
\usepackage{enumerate}
\usepackage{latexsym}
\usepackage{epsf}
\usepackage{epsfig}
\usepackage{euscript}
\usepackage[format=hang,justification=raggedright,singlelinecheck=0,labelsep=period]{caption}
\usepackage[numbers,sort&compress]{natbib}
%\usepackage[authoryear]{natbib}
\usepackage{tikz-cd}
\usepackage{booktabs}
\usepackage[hyperfootnotes=false]{hyperref}
\usepackage[ruled, lined, longend, linesnumbered]{algorithm2e}
\usepackage{musixtex}

%DO NOT PLACE ANY PACKAGES AFTER THE HYPERREF SET UP
\hypersetup{colorlinks=true,linkcolor=blue,anchorcolor=blue,citecolor=blue,filecolor=blue,urlcolor=blue,bookmarksnumbered=true,pdfview=FitB}
%\def\UrlFont{\rmfamily} % use this line for Times New Roman
\def\UrlFont{\sffamily} % use this line for CMSS

% This command allows equation arrays and similar environments to break across pages to improve text flow - use only if needed.
%\allowdisplaybreaks

% Prevent figures, tables or algorithms from using a separate page or column alone
\renewcommand{\topfraction}{0.85}
\renewcommand{\textfraction}{0.1}
\renewcommand{\floatpagefraction}{0.75}

% correct bad hyphenation here
%\hyphenation{op-tical net-works semi-C:\Program Files\MiKTeX 2.5\miktexconduc-tor}

% Extra commands or misc formatting such as page alignment or output paper-size commands
%\include{extraparameters}

% Set your personal and paper information
\SetFullName{Luis F. Vieira Damiani}
\SetThesisType{Thesis}
\SetDegreeType{Master of Science}
\SetGradMonth{July}
\SetGradYear{2018}
\SetDepartment{Computer Science}
\SetChair{Dr.~Beverly Sanders}

% If you have a cochair there two places in the ufthesis.cls file that will need to be uncommented as well.
% In the "getting personal information" section about line 630, and the "Abstract" Section around line 556.
% If you have a c-chair you must uncomment that line in userinfo.tex
%\SetCochair{John W. Carver III}

% Type your title here in all CAPS %
\SetTitle{TITLE GOES HERE}

% Commands to produce proper bullet list
\newlength{\widthOfItem}
\let\Itemize=\itemize
\let\endItemize=\enditemize
\renewenvironment{itemize}{%
	\begin{Itemize}
		\setlength{\itemsep}{0.5\baselineskip}
		\setlength{\labelwidth}{2em}
		\setlength{\listparindent}{.32in}%
		\setlength{\leftmargin}{.32in}
		\setlength{\rightmargin}{0in}
		\settowidth{\widthOfItem}{\labelitemi}
		\setlength{\labelsep}{\leftmargin-\widthOfItem}
		\renewcommand{\labelitemii}{--}
		\singlespacing}{%
	\end{Itemize}
}

% shortcuts for prime color text
\newcommand{\red}{\textcolor[rgb]{1.00,0.00,0.00}}
\newcommand{\green}{\textcolor[rgb]{0.00,1.00,0.00}}
\newcommand{\blue}{\textcolor[rgb]{0.00,0.00,1.00}}

\newtheorem{theorem}{Theorem}[section]
\newtheorem{lemma}[theorem]{Lemma}
\newtheorem{corollary}[theorem]{Corollary}
\newtheorem{definition}[theorem]{Definition}
\newtheorem{conjecture}[theorem]{Conjecture}
\newtheorem{proposition}[theorem]{Proposition}
\newtheorem{algorithm}[theorem]{Algorithm}
\newtheorem{example}[theorem]{Example}

\lstset
{
%	language=Matlab,
	basicstyle=\footnotesize,
	numbers=left,
%	stepnumber=1,
	showstringspaces=false,
%	tabsize=1,
	breaklines=true,
	breakatwhitespace=false,
}


\begin{document}

\maketitle
\makecopyright

\dedication{
	\addvspace{4.25in}
	\begin{center}
		\singlespacing
		Dedication goes here.\\
	\end{center}
}

\acknowledge{
	Acknowledgments go here.
}

\pdfbookmark[0]{TABLE OF CONTENTS}{tableofcontents}
\tableofcontents
\listoftables
%\setcounter{lofdepth}{2}
\listoffigures

% Produced list of abbreviations or symbols %
%\printindex[keylist]{KEY TO ABBREVIATIONS}{KEY TO ABBREVIATIONS}{}
%\printindex[mathlist]{KEY TO SYMBOLS}{KEY TO SYMBOLS}{%
%The list shown below gives a brief description of the major mathematical
%symbols defined in this work. For each symbol, the page number corresponds
%to the place where the symbol is first used.} %

\addtocontents{toc}{\protect\addvspace{10pt}\noindent{CHAPTER}\protect\hfill\par}{}
\phantomsection

%%%-----------List of Symbols, Nomenclature or Abbreviation--------

%% Please note: a list of Symbols, terms, acronyms, etc. is not usually the best practice.
%% More often you should simply define an abbreviation the first time it is used.
%% If you DO need to include a list like this please notice that it must be paginated manually
%% by breaking it up into page size tables. Longtable will not wrap the definition properly if
%% it extends to a second line and a similar issue is encountered when the tabbing environment
%% is used. If you have a better way of meeting the Editorial Office requirements I'd love to hear about it.

\chapter*{LIST OF SYMBOLS, NOMENCLATURE, OR ABBREVIATIONS} \addcontentsline{toc}{chapter}{LIST OF SYMBOLS}
%Start writing here. This is optional.
\singlespacing
\begin{tabular}{lp{5in}}
$\sum$ & Denotes the summation of a series of terms\\
\\
$\bigcap$ & A really big bigcap\\
\\
\end{tabular}
\doublespacing
 % optional

\begin{abstract}
	Abstract goes here.  
\end{abstract}

\chapter{INTRODUCTION}

Formal languages lend precision and flexibility to music specification because they require that musical ideas be turned into abstract symbols and stipulated explicitly.

Herein lies both the advantage and disadvantage of linguistic interaction with a computer music system.

The advantage is that formalized and explicit instructions can yield a high degree of control.

To create an imagined effect, composers need only specify it precisely.

They can easily stipulate music that would be difficult or impossible to perform by human beings.

In some cases, a linguistic specification is much more efficient than gestural input would be.

This is the case when a single command applies to a massive group of events, or when a short list of commands replaces dozens of pointing and selecting gestures.

The shell scripts of Unix operating systems are a typical example of command lists (Thompson and Ritchie 1974).

These advantages turn into a disadvantage when simple things must be coded in the same detail and with the same syntactic overhead as complicated things.

For example, with an alphanumeric language, envelope shapes that could be drawn on a screen in two seconds must be plotted out on paper by hand and transcribed into a list of numerical data to be typed by the composer.

For many tasks, graphical editors and visual programming systems, in which the user selects and interconnects graphical objects, are more effective and easier to use than their textual counterparts (see chapter 16).

Some languages are interactive; one can type individual statements and each of them is interpreted in turn.

This can occur in a concert situation, but the slow information rate of typing — not to mention the mundane stage presence of a typist — precludes this approach in fast­paced real­time music­making.

Gestural control through a musical input device is more efficient and natural.

Hence, languages for music, although important, do not answer all musical needs.

In the ideal, music languages should be available alongside other kinds of musical interaction tools.

\cite[785-786]{Roads1995}.

Textual languages are a precise and flexible means of controlling a computer music system.

In synthesis, score entry, and composition, they can liberate the potential of a music system from the closed world of canned software and preset hardware.

In the mid­-1980's it looked as if the Music N synthesis language dynasty might languish, due to the spread of inexpensive synthesizers.

In MIDI systems, however, the use of a score language is less common, since most music can be entered by other means (such as a music keyboard, notation program, or scanner).

For musicological applications that involve score analysis, however, a text­-based score representation may be optimum.

New textual languages for procedural composition continue to be developed, but there is a strong parallel trend toward interactive programs with graphical interfaces.

Instead of typing text, one patches together icons, draws envelopes, and fills in templates.

The textual language representation supporting the graphics is hidden from the user.

Early programming languages for music tended to favor machine efficiency over ease of use.

The present trend in programming has shifted from squeezing the last drop of efficiency out of a system to helping the user manage the complexity of layer upon layer of software and hardware.

The most common solution to this problem is object­-oriented programming (see chapter 2), and compositional applications are no exception to this trend (Pope 1991b).

\cite[817]{Roads1995}.

%\chapter{THEORETICAL FRAMEWORK}

In this chapter, we present and discuss the tools and methodology utilized to build \emph{Scandal}. We start discussing how to produce sound in the Java Runtime Environment, and particularly how it deals with real-time audio. We next give a formal presentation on how the domain-specific language was designed, and attempt to classify it among other programming languages and music-DSL's. Finally, we address the methodology involved in building its compiler, appending to the latter a short discussion on how a general-user interface was designed to provide an integrated development environment experience for the end user, as well as alternative command-line methods for compiling and running \emph{Scandal} programs.

\section{How \emph{Scandal} Manages Sound}

The JRE System Library provides a very convenient package of classes that handle the recording and reproduction of real-time audio, namely the \il{javax.sound.sampled} package. In it, we find two classes, \il{TargetDataLine} and \il{SourceDataLine}, that deal respectively with capturing audio data from the system's resources, and playing back buffers of audio data owned by the application. An instance of \il{SourceDataLine} provides a \il{write} method that takes three arguments: an array of bytes to written to a \il{Mixer} object, an integer offset, and an integer length. In \emph{Scandal}, we do not specify a \il{Mixer} object, hence we make use of one provided by the \il{System}. There are two main aspects of the \il{write} method that need to be addressed. Firstly, it blocks the thread in which it lives until the given array of bytes has been written, from offset to length, to the \il{Mixer} its \il{SourceDataLine} contains; secondly, if nothing is done, it returns as it has no more data to write. In order to have real-time audio, one then needs to be constantly feeding this \il{write} method with audio samples, for as long as one wants continuous sound output, even if these buffers of audio samples contain only zeros, i.e., silence. It immediately follows that one must specify exactly how many samples are sent at a time, naturally with consequences to the system's performance. This parameter is commonly referred in the industry as the \emph{vector size}. The trade-off is measured in terms of latency: a large a vector size helps slower systems perform better, or can allow more complex processing, or even increase polyphony. Latency, however, is bad for any live application, including the generation of MIDI notes, and the recording of live sound from a microphone. A good, low-compromise vector size is usually set to 512 samples, and normally these sizes will be powers of two. In order to specify the preferred vector size, as well as many other environment settings, \emph{Scandal} refers to a static class named \il{Settings}, which contains a static property \il{Settings.vectorSize}.

The aforementioned two characteristics of the \il{write} method within a \il{SourceDataLine} are managed by \emph{Scandal} by the class \il{AudioFlow}. In order to prevent \il{write} from prematurely returning, an instance of \il{AudioFlow} contains a boolean property named \il{running}, which is set to \il{true} for as long as real-time audio is desired. The fact that \il{write} blocks its thread, however, is managed by any class that contains itself an instance of \il{AudioFlow} as a property. The latter are in \emph{Scandal} the implementors of the \il{RealTimePerformer} interface, which is a contract that contains four abstract methods: \il{startFlow}, \il{stopFlow}, \il{getVector}, and \il{processMasterEffects}. The role of the \il{startFlow} is to merely embed an \il{AudioFlow} within a new \il{Thread} object and start this new thread. This guarantees the thread that manages audio is different than the main \il{Application} thread, hence resolving the thread-blocking issue. Once a new \il{Thread} is started in \emph{Java}, however, one cannot in general interrupt it. In order to stop the audio thread, we set the property \il{running} inside an \il{AudioFlow} to false via the \il{stopFlow} method, which causes the \il{write} method inside the \il{AudioFlow} to return. Hence one cannot resume an audio process in \emph{Scandal} at this point, even though doing so is perfectly possible in \emph{Java}. The reason for that is not that of a design choice, but rather the fact that the domain-specific language is at its infancy, and many important features that go beyond a proof-of-concept are yet to be implemented. The \il{getBuffer} method is called by the \il{AudioFlow} every time it needs to write another vector of audio samples. It is the responsibility then of any \il{RealTimePerformer} to timely compute the next \il{Settings.vectorSize} samples of audio data. Finally, the \il{processMasterEffects} routine is called from within \il{getVector} to further process the buffer of audio samples. This is usually done while the samples are still represented as floats, hence before converting them to raw bytes.

The constructor of an \il{AudioFlow} takes, in addition to a reference to a \il{RealTimePerformer}, a reference to an \il{AudioFormat} object. The latter is part of the \il{javax.sound.sampled} package and is how we ask the \il{AudioSystem} for a \il{SourceDataLine}. Instead of constructing \il{AudioFormat} objects, however, the \il{Settings} class contains static members \il{Settings.mono} and \il{Settings.stereo} that are instances of \il{AudioFormat} defining a mono and stereo format, respectively. In addition to a channel count argument, \il{AudioFormat} instances are constructed by specifying a sampling rate, and a bit depth (word length) for audio samples. Those are, too, static properties in the \il{Settings} class, namely \il{Settings.samplingRate} and \il{Settings.bitDepth}. Listing \ref{alg:play} gives the specifics of maintaining a \il{SourceDataLine} open inside an instance of \il{AudioFlow}. The latter implements, in turn, the \il{Runnable} interface, hence needs to override a \il{run} method. Inside this \il{run} method, we call the private \il{play} subroutine that is given below:

\begin{lstlisting}[language=Java,caption={Writing buffers of audio data inside the \il{play} subroutine.},label={alg:play}]
private void play() throws Exception {
	SourceDataLine sourceDataLine = AudioSystem.getSourceDataLine(format);
	sourceDataLine.open(format, Settings.vectorSize * Settings.bitDepth / 8);
	sourceDataLine.start();
	while (running) {
		ByteBuffer buffer = performer.getVector();
		sourceDataLine.write(buffer.array(), 0, buffer.position());
	}
	sourceDataLine.stop();
	sourceDataLine.close();
}
\end{lstlisting}

Inside the \il{play} subroutine, we acquire a \il{SourceDataLine} object from the \il{AudioSystem} with the specific format that the \il{RealTimePerformer} passed while constructing this \il{AudioFlow}. In order to \il{open} the data line, we need specify a buffer size in bytes, hence we multiply the vector size by the word length in bits, divided by eight, as there are eight bits per byte. We then \il{start} the data line and keep writing to it for as long as the \il{RealTimePerformer} maintains the \il{running} property inside its \il{AudioFlow} set to true. At each call to \il{write}, we ask the performer for a new vector. Filling the vector causes its position to advance until its length, hence the \il{position} method inside the \il{ByteBuffer} class will in fact return the length value we desire. The rest of the \il{play} subroutine simply releases resources before returning, at which point the audio thread is destroyed.

\section{How \emph{Scandal} Handles MIDI}

\section{The Structure of the Compiler}

In a broad perspective, the compilation process of \emph{Scandal}'s DSL has the following steps:

\begin{enumerate}
	\item A path to a \emph{.scandal} file is passed as an argument to the constructor of the compiler and a linker subroutine is called, in order to resolve any dependencies;
	\item The code is passed through a scanner, which removes white space and comments while converting strings of characters to tokens. Any illegal symbol will cause the scanner to throw an error, interrupting the compilation process;
	\item The tokens are parsed and converted into an abstract syntax tree, during which many tokens are discarded. If the order of the tokes does not match any of the constructs that \emph{Scandal} understands, the parser throws an error and interrupts the compilation process;
	\item The root of the AST begins the process of \emph{decorating} the tree, in which name references are resolved, types are checked, and variable slot numbers are assigned, whenever applicable. To keep track of names, a LeBlanc-Cook symbol table is kept. If types do not match, or names cannot be referenced, the offending node in the AST throws an error, aborting the compilation;
	\item Again starting from the root of the AST, each node generates its corresponding bytecode, making use of the \il{org.objectweb.asm} library as a facilitator. For any node that is a subroutine, its body is added following its declaration. No errors are thrown in this phase, and the root node returns an array of bytes containing the program's instructions in \emph{Java} bytecode format;
	\item Every \emph{Scandal} program implements the \il{Runnable} interface. After the compiler receives the program's bytecode, it dynamically loads that bytecode as a \emph{Java} class on the current (main) thread, causing the \emph{Scandal} program to be executed.
\end{enumerate}

\subsection{The Linking Process}

The main entry point to the compilation process is given by the \il{Compiler} class, whose constructor requires a path to a \emph{.scandal} file. This class contains a \il{link} routine that is called before each compilation to resolve dependencies, and which is given in Listing \ref{alg:link} below. The \il{Compiler} class has a property named \il{imports}, which is an array of paths to other \emph{.scandal} files upon which the program at hand depends. It also holds a \il{path} property, which was passed to its constructor, and which is used as an argument to \il{link}'s first call. A \emph{Scandal} program may have at its outermost scope \il{import} statements, which take a single string as a parameter, which in turn represents a path to a \emph{.scandal} file in the file system. Any code contained in the file may depend on this imported path's content. Similarly, the imported path's content may depend itself on other imports, and so on, provided there is no circularity, that is, nothing imports something that depends on itself. We may regard then the linking process as a directed graph, in which arrows point toward dependencies. Since we do not allow cycles, this is a directed acyclic graph. It may very well be the case that more than one import depend on a particular file, in which case we certainly do not want to import that code twice. In order to import each dependency exactly once in an order that will satisfy every node of the DAG that points to it, we need to somehow sort the array of imports. It is easy to see that this is no different than the problem of donning garments, in which one must have her socks on before putting her shoes, and where some items may call for no particular order, such as a watch \cite[612]{Cormen2009}. The solution for this problem is to topologically sort the array of imports. Since it is a DAG, however, that is very easily accomplished by a depth-first search of the graph, which is exactly what Listing \ref{alg:link} accomplishes recursively.

\begin{lstlisting}[language=Java,caption={The linking process of a \emph{Scandal} program.},label={alg:link}]
private void link(String inPath) throws Exception {
	if (imports.contains(inPath)) return;
	Program program = getProgram(getCode(inPath));
	for (Node node : program.nodes)
		if (node instanceof ImportStatement)
			link(((ImportStatement) node).expression.firstToken.text);
	imports.add(inPath);
}
\end{lstlisting}

The if-statement in line 2 of the \il{link} routine deals with the base case of the recursion, namely the case in which we have already discovered that vertex. If we are seeing a vertex for the first time, line 3 converts the code into an AST, so we can check for any \il{import} statements therein. That is, in turn, accomplished by the for-loop in line 4, which checks each node in the AST's outermost scope for \il{import} statements. For each one it finds, line 6 recursively calls the \il{link} routine with the path extracted from that \il{import} statement. Since any code upon which we might depend needs to appear \emph{before} our own, the first vertex that is finished needs to go in front of the list, and so on. To be precise, this is a \emph{reverse} topological order. If the chain of imports given by the user contains a cycle, then no topological order exists, and the \emph{Scandal} program will throw a runtime error. This is not ideal, and future versions of \emph{Scandal} will throw a compilation error instead. In order to do so, however, more structure needs to be added to the compiler, so that we may check for backward edges in the linking process, although this feature remains unimplemented.

\subsection{The Scanning Process}

The design of the entire complier takes full advantage of \emph{Java}'s object-oriented paradigm. In order to convert strings of characters from the input file into tokens, we first define a particular \emph{type} of token for each individual construct in the DSL. This is accomplished by the \il{Token} class, which contains a static enumeration \il{Kind}, that in turn defines a type for each string of characters the DSL understands. The constructor of \il{Token} takes a \il{Token.Kind} as input, and each instance of \il{Token} contains, in addition, a \il{text} property, which holds the particular string of characters for that token's kind, as well as other properties that are convenient when throwing errors, namely that token's line number, position within the input array of characters, position within the line, and length. The \il{Token} class also contains methods for converting strings into numbers, as well as convenience methods for determining whether the kind of a particular instance of \il{Token} belongs to a particular \emph{family} of tokens, i.e., whether a token is an arithmetic operator, or whether it is a comparison operator, and so on.

What the \il{Scanner} class accomplishes is the conversion of an array of characters into an array of instances of \il{Token}. The mechanism is conceptually very simple: we scan the input array from left to right and, whenever we see a string of characters that matches one of the DSL's constructs, we instantiate a new \il{Token} and add it to the array of tokens we hold, in order. In the process, we skip any white space found. These can be tab characters, space characters, new lines, hence \emph{Scandal}, unlike \emph{Python} or \emph{Make}, makes no syntactical use of line breaks or indentation. The only role white spaces play in a \emph{Scandal} program is that of improved readability. \emph{Scandal} also supports two kinds of comments: single-line, which are preceded by two forward slashes, and multi-line, where a slash \emph{immediately} followed by a star character initiates the comment, and a start immediately followed by a slash terminates it. Unlike \emph{Java} or \emph{Swift}, comments are not processed as documentation, and are thus completely discarded. Their only purpose is to document the \emph{.scandal} file in which they are contained. String literals in \emph{Scandal} are declared by enclosing the text between quotes, and single apostrophe are neither allowed, nor in the language's alphabet anywhere. Besides token kinds that bear syntactical relevance, there is an additional \emph{end-of-file} kind that exists for convenience, and is placed at the end of the token array right before the \il{scan} method returns. Checking for illegal characters, or combinations thereof, such as a name that begins with a number, for example, is all the checking the \il{Scanner} class does. All syntactical checking is delegated to the parsing stage of compilation.

\subsection{The Parsing Process}

The main purpose of the parsing stage is to convert the concrete syntax of a \emph{Scandal} program into an abstract syntax tree, where constructs are hierarchically embedded in one another. An instance of the \il{Parser} class is constructed by passing a reference to a \il{Scanner} object. The process is unraveled by invoking the \il{parse} method, which returns an instance of the \il{Program} class. A \il{Program} is a subclass of \il{Node}, an abstract class that provides basic structure for every node in the AST. In particular, \il{Program} is the node that lies at the root of the AST. For each construct specified by the concrete syntax of \emph{Scandal}, there is a corresponding construct specified by its abstract syntax. More often than not, the abstract construct will be simpler, sometimes with many tokens removed. The job of the parser is to facilitate the process of inferring meaning from a given program, and it does so by going, from left to right, through the array of tokens passed by the scanner and, whenever it sees a sequence of tokens that matches one of the constructs in the concrete syntax, it consumes those tokens and creates a subclass of \il{Node} that corresponds to the construct at hand. It follows, for every acceptable construct in the DSL, there is a subclass of \il{Node} that defines it. Some nodes are nested hierarchically in others, and ultimately all nodes are nested in an instance of \il{Program}, hence why the parsing stage ultimately constructs a tree.

Structuring a program hierarchically is essential for inferring the meaning of complex expressions that have some sort of precedence relation among its sub-expressions. That is the case of arithmetic operations, in which, say, multiplication has precedence over addition, and exponentiation has precedence over multiplication. As an example, \emph{Supercollider} evaluates $3 + 3 * 3$ to $18$, since it parses the expression from left to right without regard for the precedence relations among arithmetic operators. This is counterintuitive, and does not correspond to how mathematical expressions are evaluated in general. We would like, instead, the expression $3 + 3 * 3$ to evaluate to $12$, in which case we cannot take it from left to right. Rather, we must first evaluate $e_2 = 3 * 3$, \emph{then} evaluate $e_1 = 3 + e_2$. It is easy to see that, no matter how complex the expression might be, we can always represent it as a \emph{binary} tree by taking the leftmost, highest-precedence operator and splitting the expression in half at that point. We then look at each sub-expression and do the same, until we reach a leaf. Note that the AST is not, in general, a binary tree. If two operators have the same precedence, we associate from left to right, that is, $1 - 2 + 3 = (1 - 2) + 3 = 2$, which also corresponds to how mathematical expressions associate. Complex expressions are dealt in \emph{Scandal} by the \il{BinaryExpression} class, and the fact that instances of \il{BinaryExpression} may contain other instances of \il{BinaryExpression} simply means we must construct them recursively. We shall discuss in detail each syntactical construct of \emph{Scandal} in the sections that follow, along with their concrete and abstract syntax definitions, and parsing routines.

% TODO: add graph of a binary tree

\subsection{Decorating the AST}

The idea of representing a program as a tree has many advantages, chief among them being the fact we can traverse the tree to infer its meaning. This is often non-trivial, and is necessary as many constructs are name-references to other constructs, and require that we look back to how they were originally declared if we are to make sense of them. In \emph{Scandal}, every subclass of \il{Node} overrides the abstract method \il{decorate}, which in turn takes an instance of \il{SymbolTable} as an argument. The latter is a class that implements a LeBlanc-Cook symbol table \cite{Cook1983}. Several nodes in the DSL define new naming scopes, \il{Program} being the node that holds the zeroth scope. These nodes are namely those that have \il{Block}, or its subclass \il{LambdaBlock} as members. \il{IfStatement} and \il{WhileStatement} both have \il{Block} as a child node, whereas \il{LambdaLitBlock} points to a \il{LambdaBlock}, who differs from \il{Block} in that it has a return statement. \il{LambdaBlock} only exist in the context of a lambda literal expression, however instances of \il{Block}, inside if or while-statements or on their own, may exits arbitrarily, always defining new naming scopes. Every time we enter a new scope in \emph{Scandal}, we have access to variables that were declared in outer scopes, but the converse is not true. Also, every time we enter a new scope, we have the opportunity of re-declaring variables' names without the risk of clashing with names already declared in outer scopes. For each scope, we hold a hash table whose keys are the variables' names, and whose values are subclasses of the abstract type \il{Declaration}. In order to \emph{remember} as we enter new scopes, and \emph{forget} as we leave them, an instance of \il{SymbolTable} holds a \il{Stack} of name-declaration hash tables, since stacks are exactly the kind of data structure that gives us this last-in, first-out behavior. In order to trigger the whole process of decorating the AST, the \il{Compiler} class instantiates a \il{SymbolTable}, and passes that as an argument to the instance of \il{Program} that was returned by the parser. Listing~\ref{alg:compile} shows how this is done in the \il{compile} method inside \il{Program}. Since every node overrides the \il{decorate} method, this instance of \il{SymbolTable} is passed down along the entire tree. Nodes that introduce new naming scopes have the responsibility of pushing a new hash table onto the stack, then popping it before returning from the \il{decorate} method.

In addition to resolving names, the decoration process is crucial for type-checking expressions and statements in the DSL. Even though the \emph{Java} bytecode instructions are explicitly typed, languages that compile to bytecode do not need to be. That is the case of \emph{Scala} and \emph{Groovy}, in which types can be inferred, or declared explicitly. Furthermore, there is a degree of latitude to which types can actually \emph{change} in the bytecode implementation. The JVM only cares that, once a variable is stored in a certain slot number as, say, a float, that is, using the instruction \il{FSTORE}, that it be retrieved too as a float, that is, using the instruction \il{FLOAD}. It is perfectly possible to use the same slot number to, say, \il{ISTORE} an integer value. The only consistency the JVM requires is that, for as long as that slot holds an integer, the value can only be retrieved by an \il{ILOAD} instruction. This requirement naturally extends to method signatures, which are also explicitly typed in the JVM. Hence, like \emph{JavaScript}, one can theoretically change the type of a variable attached to a name after it has been declared; unlike \emph{JavaScript}, however, types have to be assigned to arguments when declaring a method, and that method signature is immutable. It is still possible to overload a method to accept multiple signatures, but overloaded names are still \emph{different} methods, with altogether different bodies. The same applies to non-primitive types, that is, types that are instances of a class in the JVM. To store or retrieve non-primitive types, we use the \il{ASTORE} and \il{ALOAD} JVM instructions, respectively. Hence it is also theoretically possible to overwrite non-primitive types. However, method signatures that take non-primitives require a fully-qualified class name, hence are immutable as above. A fully-qualified name is the name of the class, preceded by the names of the packages in which it is contained, separated by forward slashes. For \il{Compiler}, for example, we have \il{language/compiler/Compiler}.

\begin{lstlisting}[language=Java,caption={Triggering the compilation process of a \emph{Scandal} program.},label={alg:compile}]
public void compile() throws Exception {
	imports.clear();
	code = "";
	link(path);
	for (String p : imports) code += getCode(p);
	symtab = new SymbolTable(className);
	program = getProgram(code);
	program.decorate(symtab);
	program.generate(null, symtab);
}
\end{lstlisting}

\emph{Scandal} is, by design choice, strongly typed. There are many reasons for that. The main reason is that the only kind of method it supports is that of a lambda expression, even though \emph{Scandal} is not a pure functional language. These lambda expressions define themselves their own parametrized sub-types, hence a lot of what the language \emph{is} hinges on type safety. It is also a design choice to make \emph{Scandal} accessible as an entry-level language, that is, directed toward an audience interested in learning audio signal processing in more depth, without the implementation hiding inherent to the unit-generator concept. Having types explicitly defined can help inexperienced programmers better debug their code, as well as help them understand the underlying implementation of the language. Type inference is, in essence, another way of hiding implementation, which has advantages, but also drawbacks. It is notoriously difficult to report errors and debug large projects in an IDE with languages that are not strongly typed. That is certainly the case with \emph{JavaScript}, of which \emph{TypeScript} is a typed superset aimed exactly at facilitating development within an IDE. \emph{Scandal} is fully integrate into its IDE, where reporting compilation errors to the programmer is a lot more informative, hence educational, than throwing runtime errors and aborting execution. For all these reasons, type-checking is one of the main jobs the decoration process accomplished. It can become rather involved, especially when it comes to composing partial applications of lambda expressions. We shall describe the intricacies of type checking alongside each of the DSL's constructs in the sections that follow.

\subsection{Generating Bytecode}

Similarly to the decoration process, bytecode generation is triggered from the root of the AST, that is, an instance of \il{Program} received from the parser, and which has been already decorated, and passed down to every node of the tree by a common abstract method each subclass of \il{Node} overrides. In this case, this common method is called \il{generate}, and it takes two arguments. The first is an instance of \il{org.objectweb.asm.MethodVisitor}, and the second is the the decorated instance of \il{SymbolTable}. \il{MethodVisitor} is part of the ASM library, which is a convenient set of tools aimed at facilitating the generation of \emph{Java} bytecode. As the name suggests, it visits a method within the bytecode class and adds statements to it. As can be seen in line 9 of Listing~\ref{alg:compile}, a null pointer is passed to the very first call to \il{generate}, since at that point we have not created any methods in the bytecode class yet. Every \emph{Scandal} program compiles to a \emph{Java} class, which in turn implements the \il{Runnable} interface. Inside the class, there are three methods: \il{init}, where we create the method bodies of lambda literal expressions, which are always fields in the \emph{Java} class; \il{run}, which is a required override of the \il{Runnable} interface, and where we create all \emph{Scandal} local variables and statements; and \il{main}, where we instantiate the class and call \il{run}. Inside \il{Program}, the \il{generate} method creates three instances of \il{MethodVisitor}, one for each aforementioned method.

If and only if a child node is an instance of \il{LambdaLitDeclaration}, a \il{Node} used to declare a name and assign to it a lambda literal expression, this child node is passed an instance of \il{MethodVisitor} that lives inside the \il{init} method. The immediate implication of this design choice is that lambda literal expressions are always global variables in a \emph{Scandal} program, thus accessible everywhere. However, they must be declared at the outermost scope of the program, and will throw a compilation error if declared elsewhere. A similar design pattern applies to nodes that are instances of \il{FieldDeclaration}, a \il{Node} used to declare field variables in the \emph{Java} class, which in turn correspond to global variables in the \emph{Scandal} program. For both \il{LambdaLitDeclaration} and \il{FieldDeclaration} nodes, we need to add field declarations in the \emph{Java} class, which is accomplished by instantiating, for each of these nodes, a \il{org.objectweb.asm.FieldVisitor}. This is only ever done inside \il{Program} hence, as a consequence, global variables in a \emph{Scandal} program must always be declared at the outermost scope. Similarly to instances of \il{LambdaLitDeclaration}, instances of \il{FieldDeclaration} in inner scopes throw a compilation error. Every descendant of the root node that is \emph{not} an instance of \il{LambdaLitDeclaration} receives as a parameter to its \il{generate} method an instance of \il{MethodVisitor} that lives inside the \il{run} method of the \emph{Java} class. This includes instances of \il{FieldDeclaration}, which are only declared by a \il{FieldVisitor}, and whose assignment is done inside the \il{run} method, along with all other declarations and statements.

Unlike instances of \il{FieldDeclaration}, instances of \il{LambdaLitDeclaration} are marked as \emph{final} in the \emph{Java} class, hence cannot be reassigned. The reason is simple: once reassigning a variable that points to a method body, the latter may become inaccessible. In \emph{Scandal}, one can create references to lambdas inside the \il{run} method, which are not instances of \il{LambdaLitDeclaration}, that is, which do not specify a method body. These references are rather instances of the superclass \il{AssignmentDeclaration}, and can be freely reassigned, even to lambdas that have different parameters, i.e., method signatures, that that of the original assignment. Reassigning references to lambdas come allow for great code re-usability. There is a third subclass of \il{Node} which can only be used at the outermost scope, namely \il{ImportStatement}. The reason is, besides clarity and organization of \emph{Scandal} code, because the \il{link} routine inside \il{Program} only looks for import statements within the outermost scope of a program's AST. In all three such nodes, checking whether that particular instance lives in the outermost scope is a simple matter of asking the passed instance of \il{SymbolTable} whether the current scope number is zero.

\subsection{Running a \emph{Scandal} Program}

Every \il{Program} node holds an array of bytes corresponding to a binary representation of the compiled \emph{Scandal} program. This array is created right before the \il{generate} method returns. The \il{Compliler} class naturally holds a reference to an instance of \il{Program}, and utilizes the latter's \il{bytecode} property to dynamically instantiate the \emph{Scandal} program as a \emph{Java} class, that is, from an array of bytes stored in memory, rather than from a \emph{.class} in the file system. Within the IDE, a path to a \emph{Scandal} program is used to instantiate a \il{Compiler}. After calling the \il{compile} method, the resulting bytecode is used to define a subclass of \il{java.lang.ClassLoader}, namely \il{DynamicClassLoader}, which is capable of dynamically instantiating a byte array as a \emph{Java} class, as opposed to the instance returned by the static method \il{ClassLoader.getSystemClassLoader()}, which can only load classes from the file system. Once defined, we construct and instantiate the program, finally casting the result to \il{Runnable}, as illustrated in Listing \ref{alg:instance}. The \il{getInstance} method is called from the IDE by the tab that currently holds the pogram's text editor, which is an instance of \il{ScandalTab}. After retrieving the instance of \il{Runnable}, the \il{ScandalTab} simply puts is on a new \il{Thread}. Starting the thread then causes the \emph{Scandal} program to execute.

\begin{lstlisting}[language=Java,caption={Obtaining an instance of a \emph{Scandal} program.},label={alg:instance}]
public Runnable getInstance() throws Exception {
	ClassLoader context = ClassLoader.getSystemClassLoader();
	DynamicClassLoader loader = new DynamicClassLoader(context);
	return (Runnable) loader
			.define(className, program.bytecode)
			.getConstructor()
			.newInstance();
}
\end{lstlisting}

\section{The Syntax of \emph{Scandal}}

In this section, we describe in detail every syntactical construct of \emph{Scandal}. For each of them, we state their concrete and abstract syntax definitions, how one is converted into the other in the parser, as well as the particularities of type-checking and generating bytecode. We omit some constructs that, either have trivial implementations, or whose implementations are, \emph{mutatis mutandis}, identical to other constructs, in which case we describe only a representative. In the discussion that follows, terminal symbols are in all-capital letters, productions in the concrete syntax begin with a lower-case letter, and their counterparts in the abstract syntax begins with an upper-case letter. We here present terminal symbols in the syntactical context in which they appear, and a complete list of terminal symbols can be found in Sec.~\ref{sec:symbols}.

\subsection{Top-Level Productions}

At the topmost level of a \emph{Scandal} program, there are basically only two kinds of constructs that are allowed, namely declarations and statements. The legal declarations at this level are further subdivided into three: assignment declarations, field declarations, and lambda literal declarations. In the productions that follow, the star symbol represents a \emph{Kleene} star, and or-symbols and parenthesis are not tokens in the language. As a rule, terminal symbols will be given by their names, like \texttt{OR}, to avoid confusion with symbols in the language's grammar. Below are the production rules for \il{program}:

\begin{itemize}
	\item type := \texttt{KW\_INT} $|$ \texttt{KW\_FLOAT} $|$ \texttt{KW\_BOOL} $|$ \texttt{KW\_STRING} $|$ \texttt{KW\_ARRAY} $|$ \texttt{KW\_LAMBDA}
	\item declaration := assignmentDec $|$ fieldDec $|$ lambdaLitDec $|$ paramDec
	\item program := (assignmentDec $|$ fieldDec $|$ lambdaLitDec $|$ statement)$^*$
\end{itemize}

In the AST, \il{Declaration} is an abstract class that has two subclasses: \il{AssignmentDeclaration}, and \il{ParamDeclaration}. The former has two subclasses, \il{FieldDeclaration}, and \il{LambdaLitDeclaration}. The primary difference between the two subclasses of \il{Declaration} is that the latter defines a type and a name without binding any value to that name at the time of declaration, while the former requires that some expression be given at the moment the variable is declared. It follows every variable declaration in \emph{Scandal} must be initialized, except when they are parameters of a lambda literal, in which case they actually cannot be initialized. A \il{program} can contain any number of \il{assignmentDec}, \il{fieldDec}, or \il{lambdaLitDec}, in any order, while a \il{paramDec} only exist in the context of a lambda literal. As will be seen below, every \il{declaration} begins with a type token, hence FIRST(\il{declaration}) = \il{type}. The abstract syntax of \il{Program} is then:

\begin{itemize}
	\item Declaration := AssignmentDeclaration $|$ FieldDeclaration
	\item Declaration := LambdaLitDeclaration $|$ ParamDeclaration
	\item Program := (AssignmentDeclaration $|$ FieldDeclaration)$^*$
	\item Program := (LambdaLitDeclaration $|$ Statement)$^*$
\end{itemize}

The parsing routine for \il{program} is very simple, and constructs an instance of \il{Program} by checking whether the next token in the array of tokens produced by the scanner is in the FIRST set of \il{declaration}. If it is, we attempt to construct an instance or subclass of \il{AssignmentDeclaration}, consuming in the process all the tokens therein. If not, we attempt to construct a subclass of the abstract type \il{Statement}. That is done much that same way, by looking at the set FIRST(\il{statement}). Listing \ref{alg:program} shows how a concrete \il{program} is converted into a \il{Program} node in the AST.

\begin{lstlisting}[language=Java,caption={Parsing topmost-level constructs in \emph{Scandal}.},label={alg:program}]
public Program parse() throws Exception {
	Token firstToken = token;
	ArrayList<Node> nodes = new ArrayList<>();
	while (token.kind != EOF) {
		if (token.isDeclaration()) nodes.add(assignmentDeclaration());
		else nodes.add(statement());
	}
	matchEOF();
	return new Program(firstToken, nodes);
}
\end{lstlisting}

In Listing \ref{alg:program}, we construct an instance of \il{Program} by first creating an array of nodes. These nodes, however, must be either a subclass of \il{AssignmentDeclaration}, or a subclass of the abstract class \il{Statement}. Line 5 checks whether the next unconsumed token is in the FIRST set of a declaration. If so, further parsing is delegated to the \il{assignmentDeclaration} routine. If not, the only other option is that the next token initiates a statement, and parsing thereof is delegated to the \il{statement} routine in line 6. Nodes are added in the exact order in which they appear in the \emph{Scandal} program, regardless whether they are declarations or statements. An end-of-file token was included in the scanning process for convenience, and here we make use of it by checking the next available token against the \texttt{EOF} kind. As soon as we find it, we know we have reached the end of the token array, and can thus stop looking for declarations and statements. If we were expecting a particular token, but \texttt{EOF} appeared prematurely, we throw an error.

Inside an instance of \il{Program}, type-checking is completely delegated to each node in the node array. More precisely, inside the \il{decorate} routine, we iterate over the node array and, for each node, we call \il{node.decorate}, passing along the symbol table instantiated by the compiler. Generating bytecode, on the other hand, is a lot more complex, since we need to provide the overall structure for the entire \emph{Java} class. That is accomplished inside the \il{generate} method by creating an instance of \il{org.objectweb.asm.ClassWriter}. The latter, which we call \il{cw}, manages the creation of the \emph{Java} class itself, including the generation of the byte array used to instantiate and run the \emph{Scandal} program. In particular, we set the JRE to version 1.8, make the access to the class \il{public}, and define it as a subclass of \il{java/lang/Object} that implements the \il{java/lang/Runnable} interface. We then create three instances of \il{MethodVisitor} by calling \il{cw.visitMethod}, one for each method in the \emph{Java} class. The methods are namely \il{init}, \il{run}, and \il{main}. In \il{init}, we basically go through the node array and, if the particular node is an instance of \il{LambdaLitDeclaration}, we call \il{node.generate}, passing the appropriate instance of \il{MethodVisitor} and our symbol table as parameters. What the \il{generate} method does inside a \il{LambdaLitDeclaration} is somewhat complicated, and we defer its explanation to the moment we discuss the \il{LambdaLitExpression} class. Before visiting \il{run}, we go once again over all nodes in the node array and, if they are either an instance of \il{LambdaLitDeclaration} or an instance of \il{FieldDeclaration}, we call \il{cw.visitField}. This method creates fields in the \emhp{Java} class, which correspond to \il{field} variables in the \emph{Scandal} program. Every field is marked as \il{static}, since we make no use of \emph{Java}'s object-oriented paradigm. In addition, lambda fields are marked as \il{final}, as previously discussed, and we take the opportunity to pass \il{cw} along to the instances of \il{LambdaLitExpression} for which we are creating field declarations, and ask them to create a method body for the lambda literal expression. This is accomplished inside each lambda literal expression by an overloaded \il{generate} method, which takes, instead of a \il{MethodWriter}, the instance of \il{ClassWriter}, namely \il{cw}, and uses that to create its own \il{MethodWriter}, which will correspond to the lambda's method. The instances of \il{LambdaLitExpression} are accessed through the \il{lambda} property inside the \il{LambdaLitDeclaration}, and the particularities of creating method bodies for lambdas will be discussed momentarily. The next step is to add a body for the \emph{Java} class' \il{run} method. To do so, we go yet once more over the array of nodes, and this time we generate any node that is \emph{not} an instance of \il{LambdaLitDeclaration}, for obvious reasons. We do visit instances of \il{FieldDeclaration}, since \il{cw.visitField} only created the field, but never assigned any value to it. Since we only allow unassigned declarations in \emph{Scandal} when declaring lambda parameters, we have something to assign to that field, and \il{generate} inside \il{FieldDeclaration} takes care of that.

\begin{lstlisting}[language=Java,caption={Using the ASM framework to construct a \il{main} method.},label={alg:main}]
private void addMain(ClassWriter cw, SymbolTable symtab) {
	MethodVisitor mv =
		cw.visitMethod(ACC_PUBLIC + ACC_STATIC, "main", "([Ljava/lang/String;)V", null, null);
	mv.visitTypeInsn(NEW, symtab.className);
	mv.visitInsn(DUP);
	mv.visitMethodInsn(INVOKESPECIAL, symtab.className, "<init>", "()V", false);
	mv.visitMethodInsn(INVOKEVIRTUAL, symtab.className, "run", "()V", false);
	mv.visitInsn(RETURN);
	mv.visitMaxs(0, 0);
}
\end{lstlisting}

Finally, we visit the \il{main} method in the \emph{Java} class, which is shown in Listing \ref{alg:main}. This is the standard \il{main} method in \emph{Java}, which is always \il{public} and \il{static}, takes an array of strings, and returns nothing. Line 3 uses \il{cw} to create an instance of \il{MethodVisitor}, namely \il{mv}, with exactly these properties. Bytecode syntax for method signatures is given by a parenthesized list of argument types, followed by the return type. Hence a \il{void} method that takes a \il{String[]} in \emph{Java} becomes \il{([Ljava/lang/String;)V}, where the left bracket means we have an array of whatever type follows, and the colon separates argument types. Naturally, \il{java/lang/String} is a string, and \il{V} stands for the \il{void} type. The JVM is stack-based, so in line 5 we create a new instance of the \emph{Java} class, whose name is stored in our symbol table, and leave it on top of the stack. In line 6 we duplicate whatever is on top of the stack, since we will need to use our newly created \emph{Java} class twice, namely to call on it \il{init}, and then \il{run}. These two calls are made in lines 6 and 7, respectively. Notice both method signatures take no arguments and return nothing, hence are equivalent to \il{()V} in bytecode. Finally, we add a return statement to the \il{main} method's body, which is omitted in \il{void} \emph{Java} methods, but required in bytecode. A bytecode method requires that we compute the maximum number of elements the stack will have, as well as the total number of local variables in the method. ASM does that for us, and we asked it to do so by passing a \il{ClassWriter.COMPUTE\_FRAMES} as an argument while constructing \il{cw}. The two arguments to \il{mv.visitMaxs} are the maximum stack size, and the total local variables. We pass zeros since we are not computing them, but the call must be made nonetheless.

\subsection{Subclasses of \il{Declaration}}

The class \il{Declaration} is an abstract type that extends \il{Node} by adding three instance variables, namely a \il{Token} to hold the name we are declaring, an integer to hold its slot number, and a boolean property to distinguish whether this is a field or not. Slot numbers are not necessary for fields, hence are only used in the context of the \emph{Java} class' \il{run} method. \il{Declaration} branches out into two non-abstract subclasses, \il{ParamDeclaration} and \il{AssignmentDeclaration}. The latter has itself two other subclasses, \il{FieldDeclaration} and \il{LambdaLitDeclaration}. As discussed above, the difference is that \il{ParamDeclaration} only occurs inside a \il{LambdaLitDeclaration}; \il{FieldDeclaration} and \il{LambdaLitDeclaration} only occur at the outermost scope; and \il{AssignmentDeclaration} occurs anywhere. Below are the production rules for the concrete syntax of declarations in \emph{Scandal}.

\begin{itemize}
	\item paramDeclaration := type \texttt{IDENT}
	\item assignmentDeclaration := type \texttt{IDENT} \texttt{ASSIGN} expression
	\item fieldDeclaration := \texttt{KW\_FIELD} assignmentDeclaration
	\item lambdaAssignment := \texttt{KW\_LAMBDA} \texttt{IDENT} \texttt{ASSIGN} (paramDeclaration)$^*$
	\item lambdaLitDeclaration := lambdaAssignment lambdaLit
	\item lambdaLitDeclaration := lambdaAssignment lambdaBlock
\end{itemize}

As shown in line 5 of Listing \ref{alg:program}, we parse declarations by looking at the very first token at hand, since the FIRST set of 




\begin{lstlisting}[language=Java,caption={},label={alg:assign}]
public AssignmentDeclaration assignmentDeclaration() throws Exception {
	boolean isField = token.kind == KW_FIELD;
	if (isField) consume();
	Token firstToken = consume();
	Token identToken = match(IDENT);
	match(ASSIGN);
	Expression e = expression();
	if (e instanceof LambdaLitExpression)
		return new LambdaLitDeclaration(firstToken, identToken, (LambdaLitExpression) e);
	if (isField) return new FieldDeclaration(firstToken, identToken, e);
	return new AssignmentDeclaration(firstToken, identToken, e);
}
\end{lstlisting}

\begin{lstlisting}[language=Java,caption={},label={alg:param}]
public ParamDeclaration paramDeclaration() throws Exception {
	Token firstToken = consume();
	Token identToken = match(IDENT);
	return new ParamDeclaration(firstToken, identToken);
}
\end{lstlisting}

\subsubsection{The \il{ParamDeclaration} Class}

\il{ParamDeclaration} is basically an implementation of \il{Declaration}. It adds no new properties, thus consisting of basically a type \il{Token} and an identifier \il{Token}. Since it is not abstract, it must override \il{decorate} and \il{generate}, the two abstract methods in \il{Node} that provide functionality to all nodes in the AST.

\subsubsection{The \il{AssignmentDeclaration} Class}

\subsubsection{The \il{FieldDeclaration} Class}

\subsubsection{The \il{LambdaLitDeclaration} Class}

%\chapter{LITERATURE REVIEW} % TODO: see Howe 1975

Arguably the first notable attempt to design a programming language with an explicit intent of processing sounds and making music was that of \emph{Music I}, created in 1957 by Max Mathews. The language was indented to run on an IBM 704 computer, located at the IBM headquarters in New York City. The programs created there were recorded on digital magnetic tape, then converted to analog at Bell Labs, where Mathews spent most of his career as an electrical engineer. \emph{Music I} was capable of generating a single waveform, namely a triangle, as well as assigning duration, pitch, amplitude, and the same value for decay and release time. \emph{Music II} followed a year later, taking advantage of the much more efficient IBM 7094 to produce up to four independent voices chosen from 16 waveforms. With \emph{Music III}, Mathews introduced in 1960 the concept of a \emph{unit generator}, which consisted of small building blocks of software that allowed composers to make use of the language with a lot less effort and required background. In 1963, \emph{Music IV} introduced the use of macros, which had just been invented, although the programming was still done in assembly language, hence all implementations of the program remained machine-dependent. With the increasing popularity of Fortran, Mathews designed \emph{Music V} with the intent of making it machine-independent, at least in part, since the unit generators' inner loops were still programmed in machine language. The reason for that is the burden these loops imposed on the computer \cite[15-17]{Roads1980}.

\section{Software Synthesis Languages}

Since Mathews' early work, much progress has been made, and a myriad of new programming languages that support sound processing, as well as domain-specific languages whose sole purpose is to process sounds or musical events, have surfaced. In \cite{Roads1995}, we see an attempt to classify these languages according to the specific aspect of sound processing they perform best. The first broad category described is that of \emph{software synthesis languages}, which compute samples in non-real-time, and are implemented by use of a text editor with a general purpose computer. The \emph{Music N} family of languages consist of software synthesis languages. A characteristic common to all software synthesis languages is that if a toolkit approach to sound synthesis, whereby using the toolkit is straightforward, however customizing it to fulfill particular needs often require knowledge of the programming language in which the toolkit was implemented. This approach provides great flexibility, but at the expense of a much steeper learning curve. Another aspect of software synthesis languages is that they can support an arbitrary number of voices, and the time complexity of the algorithms used only influences the processing time, not the ability to process sound at all, as we see with real-time implementations. As a result of being non-real-time, software synthesis languages usually lack controls that are gestural in nature. Yet, software synthesis languages are capable of processing sounds with a very fine numerical detail, although this usually translates to more detailed, hence verbose code. Software synthesis languages, or non-real-time features of a more general-purpose language, are sometimes required to realize specific musical ideas and sound-processing applications that are impossible to realize in real time \cite[783-787]{Roads1995}.

Within the category of software synthesis languages, we can further classify those that are \emph{unit generator languages}. This is exactly the paradigm originally introduced by \emph{Music III}. In them, we usually have a separation between an orchestra section, and a score section, often given by different files and sub-languages. A unit generator is more often than not a built-in feature of the language. Unit generators can generate or transform buffers of audio data, as well as deal with how the language interacts with the hardware, that is, provide sound input, output, or print statements to the console. Even though one can usually define unit generators in terms of the language itself, the common practice is to define them as part of the language implementation itself. Another characteristic of unit generators is that they are designed to take as input arguments the outputs of other unit generators, thus creating a signal flow. This is implemented by keeping data arrays in memory which are shared by more than one UG procedure by reference. The score sub-language usually consists of a series of statements that call the routines defined by the orchestra sub-language in sequential order, often without making use of control statements. Another important aspect of the score sub-language is that it defines function lookup tables, which are mainly used to generate waveforms and envelopes. When \emph{Music N} languages became machine-independent, function generating routines remained machine-specific for a period of time, due to performance concerns. On the other hand, the orchestra sub-language is where the signal processing routines are defined. These routines are usually called instruments, and basically consist of new scopes of code where built-in functions are dove-tailed, ultimately to a unit generator that outputs sound or a sound file \cite[787-794]{Roads1995}.

The compilation process in \emph{Music N} languages consists usually of three passes. The first pass is a preprocessor, which optimizes the score that will be fed into the subsequent passes. The second pass simply sorts all function and instrument statements into chronological order. The third pass then executes each statement in order, either by filling up tables, or by calling the instrument routines defined in the orchestra. The third pass used to be the performance bottleneck in these language implementations, and during the transition between assembly and Fortran implementations, these were the parts that remained machine-specific. Initially, the output of the third pass consisted of a sound file, but eventually this part of the compilation process was adapted to generate real-time output. At that point, defining specific times for computing function tables became somewhat irrelevant.

In some software synthesis languages, the compiler offers hooks in the first two passes so that users can define their own sound-processing subroutines. In any cases, these extensions to the language were given in an altogether different language. With \emph{Common Lisp Music}, for example, one could define the data structures and control flow in terms of Lisp itself, whereas \emph{MUS10} supported the same features by accepting Algol code. In \emph{Csound}, one can still define control statements in the score using Python. Until \emph{Music IV} and its derivatives, compilation was sample-oriented. As an optimization, \emph{Music V} introduced the idea of computing samples in blocks, where audio samples maintained their time resolution, but control statements could be computed only once per block. Of course, if the block size is one, than we compute control values for each sample, as in the sample-oriented paradigm. Instead of defining a block size, however, one defines a control rate, which is simply the sampling rate times the reciprocal of the block size. Hence a control rate that equals the sampling rate would indicate a block size of one. With \emph{Cmusic}, for instance, we specify the block size directly, a notion that is consistent with the current practice of specifying a vector size in real-time implementations. The idea of determining events in the language that could be computed at different rates required some sort of type declaration. In \emph{Csound}, these are given by naming conventions: variables whose names start with the character `a' are audio-rate variables, `k' means control rate, and `i'-variables values are computed only once per statement. \emph{Csound} also utilizes naming conventions to determine scopes, with the character `g' indicating whether a variable is global \cite[799-802]{Roads1995}.

\section{Real­-Time Synthesis Control Languages}

Some of the very first notable attempts to control the real-time synthesis hardware were made at the \emph{Institut de Recherche et Coordination Acoustique/Musique} in the late seventies. Many of these early attempts made use of programming languages to drive the sound synthesis being carried out by a dedicated DSP. At first, most implementations relied on the concept of a \emph{fixed-function} hardware, which required significantly simpler software implementations, as the latter served mostly to control a circuit that had an immutable design and function. An example of such fixed-function implementations would be an early frequency-modulation synthesized, which contained a dedicated DSP for FM-synthesis, and whose software implementation would only go as far as controlling the parameters thereof. Often, the software would control a chain of interconnected dedicated DSP's, which would in turn produce envelopes, filters, and oscillators. The idea of controlling parameters through software, while delegating all signal processing to hardware, soon expanded beyond the control of synthesis parameters, and into the sequencing of musical events, like in the New England Digital Synclavier. Gradually, these commercial products began to offer the possibility of changing how exactly this components were interconnected, what is called a \emph{variable-function} DSP hardware. Interconnecting these components through software became commonly called \emph{patching}, as an analogy to analog synthesizers. The idea of patching brought more flexibility, but imposed a steeper learning curve to musicians. Eventually, these dedicated DSP's were substituted by general-purpose computers, wherein the entire chain of signal processing would be accomplished via software \cite[802-804]{Roads1995}.

Commonly in a fixed-function implementation there is some sort of front panel with a small LCD, along with buttons and knobs to manage user input. In the case of a keyboard instrument, there is naturally a keyboard to manage this interaction, as well. The purpose of the embedded software is then to communicate user input to an embedded system which contains a microprocessor and does the actual audio signal processing, memory management, and audio input/output. All software is installed in some read-only memory, including the operating system. With the creation of the \emph{Musical Instrument Digital Interface} standard in 1983, which was promptly absorbed my most commercial brands, the issue of controlling sound synthesis hardware transcended the interaction with keys, buttons, and sliders, and became a matter of software programming, as one could easily communicate with dedicated hardware, by means of a serial interface, MIDI messages containing discrete note data, continuous controller messages, discrete program change messages, as well as system­-exclusive messages. As a trend, many MIDI libraries were written at the time for general-purpose programming languages such as APL, Basic, C, Pascal, Hypertalk, Forth, and Lisp. In addition, most descendants of the \emph{Music N} family of languages began to also support MIDI messages as a way to control dedicated hardware \cite[804-805]{Roads1995}.

The implementation of a software application to control variable-function DSP hardware is no mundane task, as it requires knowledge of digital signal processing, in addition to programming in a relatively low level language. Dealing with issues of performance, memory management, let alone the mathematics required to process buffers of audio samples, often imposes an unsurmountable burden to musicians. Many solutions were invented in order to work around this difficulties, including the use of graphic elements and controllers, but ultimately it was the concept of a unit generator, borrowed from software synthesis languages, that most influenced the creation of higher-level abstractions that were more suitable for musicians. This is notably the case of the \emph{4CED} language, which was developed at IRCAM in 1980, and owed greatly to \emph{Music IV} and \emph{Music V}. The resemblance extended as far as to comprise a separate orchestra sub-language for patching unit generators, a score sub-language, and a third command sub-language for controlling effects in real-time, as well as to link both orchestra and score to external input devices such as buttons and potentiometers. The hardware these languages drove was IRCAM's 4C synthesizer. The result of nearly a decade of research at IRCAM culminated in \emph{Max}, a visual programming language that remains to this day one of the most important real-time tools for musicians. \emph{Max}, which will later be discussed in more detail, eventually transcended its hardware DSP and implemented itself in C the sound-generating routines. But that was not until the 2000's, ten years after it became a commercial software application, independent of IRCAM \cite[805-806]{Roads1995}.

\begin{example}
	\cite[809]{Roads1995}
	\emph{Music 1000} is a descendant of the \emph{Music N} family of languages that was designed to drive the Digital Music Systems DMX­1000 signal processing computer, in which we can clearly observe the unit-generator concept in action:
	\begin{algorithm}
		fnctn func1, 512, fourier, normal, 1, 1000\\
		instr 1\\
		\Indp
			kscale amp, knob1, 0, 10000\\
			kscale freq, knob2, 20, 2000\\
			oscil x8, \#func1, amp, freq\\
			out x8\\
		\Indm
		endin\\
		\caption{\emph{Music 1000} algorithm that produces a sine wave.}
	\end{algorithm}
	In the code above, a \emph{fnctn} statement assigns to variable \emph{func1} an array of 512 samples using a \emph{fourier} series of exactly one harmonically-related sine, whose (trivial) sum is \emph{normal}-ized. The amplitude of 1000 is then meaningless, but a required argument. In fact, \emph{func1} takes a variable number of arguments, where for each harmonic partial, the user specifies a relative amplitude. The block that follows defines an instrument, in which the unit generator \emph{oscil} takes as arguments the output of three other unit generators, which are respectively the wavetable previously computed, as well as amplitude and frequency parameters, whose values are in turn captured by two knobs attached to the machine. The knobs produce values between 0 and 1, and the subsequent arguments to \emph{kscale} are scaling parameters. Finally, \emph{out} is a unit generator that connects the output of \emph{oscil} to the digital-to-analog converter.
\end{example}

\section{Music Composition Languages}

Between the 1960's and the 1990's, many programming languages were devised to aid music composition. As a noticeable trend, one can define two categories among those languages, namely those that are \emph{score input languages}, and those that are \emph{procedural languages}. The main difference between the two categories is that, in the former, some representation of a musical composition is already at hand, hence score input languages provide a way to encode that information. This could be a score, a MIDI note list, or even some graphical representation of music. In the latter category, the language provides, or helps define procedures that are used to generate musical material, a practice that is often called \emph{algorithmic} music composition. One outstanding characteristic of score input languages is how verbose and complex they can become, depending on the musical material they are trying to represent. This difficulty influenced the devising of many alternatives to textual programming languages, such as the use of scanners in the late 1990's by Neuratron's \emph{PhotoScore}, an implementation which was predicted by composer Milton Babbitt as early as in 1965. Before the advent of MIDI, however, programming languages were indeed the user interface technology of choice, or lack thereof, to design applications meant for analyzing, synthesizing, and printing musical scores. With the widespread adoption of the MIDI standard in the mid-1980's, whereby one can input note events by performing on a MIDI instrument, combined with the advancements in graphical user interfaces of the mid-1990's, the creation and maintenance of score input languages has faced a huge decline. What is even worse, the paradigm of a musical score is itself inadequate for computer music synthesis, in that a score is more often than not a very incomplete representation of a musical piece, often omitting a great deal of information. It is the job of a musical performer to provide that missing information. In this sense, \emph{procedural languages} are much better suited for computer performance, but that comes at the cost of replacing the score paradigm altogether \cite[811-813]{Roads1995}.

In 2018, a few \emph{score input languages} remain, despite the vast predominance of graphical user interfaces as a means to input notes to a score. \emph{MusiXTex} is a surviving example that compiles to \LaTeX, which in turn compiles to PDF documents. It was created in 1991 by Daniel Taupin. The language has such unwieldy syntax, that often a preprocessor is required for more complex scores. One famous such processors is \emph{PMX}, a \emph{FORTRAN} tool written by Don Simons in the late 1990's. Another was \emph{MPP}, which stands for MusiXTex Preprocessor, created by Han-Wen Nienhuys and Jan Nieuwenhuizen in 1996, and which eventually became \emph{LilyPond}, arguably the most complete surviving score input language today. \emph{LilyPond} has a much simpler syntax than that of \emph{MusiXTex}, however not nearly as simple as \emph{ABC} music notation, a language that much resembles \emph{Musica} and which is traditionally used in music education contexts. A package written by Guido Gonzato is available in \LaTeX which can produce simple scores in \emph{ABC} notation. Its simplicity comes, however, at the expense of incompleteness. Finally, it is worthwhile to mention a music-notation specific standard that has emerged in the mid-2000's, namely the \emph{MusicXML} standard. Heavily influenced by the industry, it was initially meant as an object model to translate scores between commercial applications where the score input method was primarily graphical, and whose underlying implementation was naturally object-oriented. \emph{MusicXML} is extremely verbose, and borderline human-readable. It is, however, very complete, to the point of dictating what features an object-oriented implementation should comprise in order to be aligned with the industry standards. In recent years, many rumors have surfaced to make \emph{MusicXML} an Internet standard, such as that of \emph{Scalable Vector Graphics}, however nothing concrete has been established.

\begin{example} \label{ex:musixtex}
	\cite[812]{Roads1995}
	\emph{Musica} was developed at the Centro di Sonologia Computazionale in Padua, Italy, and is particularly interesting in its interpreter compiles programs into \emph{Music V} note statements. 
	
	\begin{algorithm}
		4'AGAG / 4.A8G2E / 4DDFD / 2ED\\
		\caption{\emph{Musica} algorithm that creates a simple melodic line.}
	\end{algorithm}
	
\noindent In the code above, all numbers indicate note duration, that is, 4 is a quarter-note, 8 is an eighth-note, and 2 is half-note, with dots indicating dotted durations. The letters indicate pitch, and the apostrophe indicates octave such that '$A = 440$Hz, and ''$A = 880$Hz. Finally, the slash indicates a measure. The code below, on the other hand, shows an example of the very same musical material expressed in \emph{MusiXTex}. One can immediately notice the difference in implementation by the sheer amount of code required to express basically the same symbols.
	
	\begin{algorithm}
		\textbackslash{begin}\{music\}\\
		\Indp
			\textbackslash{generalmeter}\{\textbackslash{meterfrac44}\}\\
			\textbackslash{startextract}\\
			\textbackslash{Notes} \textbackslash{qu}\{h g h g\} \textbackslash{en} \textbackslash{bar}\\
			\textbackslash{Notes} \textbackslash{qup}\{h\} \textbackslash{cu}\{g\} \textbackslash{hu}\{e\} \textbackslash{en} \textbackslash{bar}\\
			\textbackslash{Notes} \textbackslash{qu}\{d d f d\} \textbackslash{en} \textbackslash{bar}\\
			\textbackslash{Notes} \textbackslash{hu}\{e d\} \textbackslash{en}\\
			\textbackslash{endextract}\\
		\Indm
		\textbackslash{end}\{music\}\\
		\caption{\emph{MusiXTex} algorithm whose output is shown in Fig.~\ref{doremi}.}
	\end{algorithm}
	
\noindent In the snippet above, many commands, despite verbose, are quite self-explanatory. Some others, however, are not. The \textbackslash{qu} command means a quarter-note with a stem pointing upward, whereas the \textbackslash{Notes} command actually means how notes should be spaced. The more capital letters, the more spacing between the notes, that is, \textbackslash{NOTes} is more spaced out than \textbackslash{NOtes}. Finally, in addition to supporting the same apostrophes as \emph{Musica} for defining octave, \emph{MusiXTex} also supports other letters, as well as capitalizations thereof. In the example above, we have $h = 440$Hz, whereas $a = 220$Hz.
\end{example}

\begin{figure}[h] \label{doremi}
	\begin{center}
	\begin{music}
		\generalmeter{\meterfrac44}
		\startextract
		\Notes \qu{h g h g} \en \bar
		\Notes \qup{h} \cu{g} \hu{e} \en \bar
		\Notes \qu{d d f d} \en \bar
		\Notes \hu{e d} \en
		\endextract
	\end{music}
	\end{center}
	\caption{Typesetting music with \emph{MusiXTex}.}
\end{figure}

One of the greatest contributions of \emph{procedural composition languages} to the field of music composition is arguably the concept of algorithmic composition, in particular when the realization of the musical algorithm is not restricted to human performers. In such circumstances, the composer is capable of exploring the full extent of musical ideas a computer can reproduce. Naturally, the composer must often trade off the ability to represent those ideas via a score, in which case the algorithm itself becomes the representation. If, on one hand, reading music from an algorithm is somewhat unfamiliar to most musicians, the representation is nonetheless formal, concise, and consistent. Furthermore, it lends itself to be analyzable a much larger apparatus of analytical techniques and visualization tools, hence is equally beneficial a representation to music theorists. A machine is capable of representing all sorts of timbres, metrics, and tunings that humans cannot, but it needs to be told exactly what to do. Unlike a human performer, who interprets the composer's intents, a purely electro-acoustic algorithmic composition must address a human audience without relying on a middle-man. Hence the programming language of choice becomes an invaluable tool for the composer. In addition to all that, another important aspect of algorithmic composition is how it is capable of transforming the decision-making process of a composer. Instead of making firm choices at the onset of a musical idea, a composer can \emph{prototype} many possible outcomes of that idea before deciding. One example is assigning random numbers to certain parameters, this postponing the decision making until more structure has been added to the composition. In fact, this postponing may be final, thus an algorithmic composition may be situated within a whole spectrum of determinism. A fully stochastic piece fixes no parameter, as opposed to a fully deterministic composition. Some of the notable techniques of electro-acoustic music composition also include spatialization, where the emission of sounds through speakers positioned at specific spatial locations constitutes a major musical dimension in a composition; spectralism, where the spectral content of sounds are manipulated by an algorithm; processing sound sources in real time, very often capturing a live performance on stage; and sonification of data nor originally conceived as sound \cite[813]{Roads1995}.

\section{Libraries}

Many domain-specific languages that deal with sound synthesis, processing, and music composition are \emph{extensible} in the sense that they provide a hook for code written in the implementation language to be executed in the context of the DSL. This feature can render a DLS a lot more flexible, at the expense of annulling the very purpose of the DSL, which can be a good trade-off if the latter's implementation is incomplete. An early example would be \emph{Music V}, which could accept user-written subroutines in \emph{Fortran}. \emph{Music 4C} had its instruments written in \emph{C}, and \emph{Cscore} was a \emph{C}-embedding of \emph{Cmusic}. Other examples are \emph{MPL}, which could accept routines written in \emph{APL}, and \emph{Pla}, whose first version was embedded in \emph{Sail}, and whose second version was embedded in \emph{Lisp}. In the particular case of \emph{Lisp}, embeddings include \emph{MIDI-LISP}, \emph{FORMES}, \emph{Esquisse}, \emph{Lisp Kernel}, \emph{Common Music}, \emph{Symbolic Composer}, \emph{Flavors Band}, and \emph{Canon}. \emph{Music Kit} was embedded in the object-oriented \emph{Objective-C} \cite[814]{Roads1995}.

Besides domain-specific languages, a variety of libraries exist for general-purpose programming languages that also deal with aspects of sound synthesis, processing, and music composition. In languages like \emph{Haskell}, these libraries may carry such syntactical weight, with so many specifically-defined symbols, that they do in fact resemble more a DSL that a library, even though such terming would not be technically correct.

%\include{tex/chapter4}

\appendix
\clearpage
\addtocontents{toc}{\protect\addvspace{10pt}\noindent{APPENDIX}\protect\hfill\par}{}

%\chapter{}

%\clearpage
%\thispagestyle{plain}
%\begin{landscape}
%\begin{figure}
%\begin{center}
%\includegraphics[width=6in]{images/LaTeX2e_logo.eps}
%\caption{\LaTeX 2\ensuremath{\epsilon.} logo}\label{biglogo}
%\end{center}
%\end{figure}
%\end{landscape}

\begin{table}[htbp]
	\caption{Unit­-generator ­based software synthesis languages \cite[789-790]{Roads1995}.}
	\centering
	\vspace{12pt}
	\begin{tabular}{ *{5}{l} }
		\hline
		Application & Year & Authors & Platform & Language \\
		\hline
		Music III & 1960 & M. Mathews & IBM 7090 & Assembler \\
		Music IV & 1963 & M. Mathews & IBM 7094 & Macro assembler \\
		& & J. Miller & & \\
		Music IVB & 1965 & G. Winham & IBM 7094 & Macro assembler \\
		& & H. Howe & & \\
		Music V & 1966 & M. Mathews & GE 645 & Fortran IV \\
		& & J. Miller & & \\
		MUS10 & 1966 & J. Chowning & DEC PDP­-10 & PDP­-10 assembler \\
		& & D. Poole & & \\
		& & L. Smith & & \\
		MUSIGOL & 1966 & D. MacInnes & Burroughs 5500 & Burroughs Algol \\
		& & W. Wulf & & \\
		& & P. Davis & & \\
		Music 4BF & 1967 & H. Howe & IBM 360 & Fortran II and \\
		& & G. Winham & & BAL assembler \\
		Music 360 & 1969 & B. Vercoe & IBM 360 & BAL assembler \\
		Music 7 & 1969 & H. Howe & Xerox XDS Sigma 7 & Assembler \\
		TEMPO & 1970 & J. Clough & IBM 360 & BAL assembler \\
		B6700 Music V & 1973 & B. Leibig & Burroughs 6700 & Fortran and Algol \\
		Music 11 & 1973 & B. Vercoe & DEC PDP-­11 & Macro-­11 assembler \\
		& & S. Haflich & & \\
		& & R. Hale & & \\
		& & H. Howe & & \\
		MUSCMP & 1978 & Tovar & Foonly 2 & FAIL assembler \\
		& & & DEC PDP­-10 & \\
		Cmusic & 1980 & F. R. Moore & DEC VAX­-11 & C \\
		& & D. G. Loy & & \\
		Cmix & 1984 & P. Lansky & DEC PDP-­11 & C \\
		Music 4C & 1985 & S. Aurenz & DEC VAX­-11 & C \\
		& & J. Beauchamp & & \\
		& & R. Maher & & \\
		& & C. Goudeseune & & \\
		Csound & 1986 & B. Vercoe & DEC VAX­-11 & C \\
		& & R. Karstens & & \\
		Music 4C & 1988 & G. Gerrard & Macintosh & C \\
		Common Lisp Music & 1991 & W. Schottstaedt & NeXT & Common Lisp \\
		\hline
	\end{tabular}
\end{table}

\begin{table}[htbp]
	\caption{Unit­-generator ­based languages for control of real-­time DSP \cite[807-808]{Roads1995}.}
	\centering
	\vspace{12pt}
	\begin{tabular}{ *{6}{l} }
		\hline
		Application & Year & Authors & Platform & Language & DSP \\
		\hline
		4B & 1978 & D. Bayer & DEC LSI­-11 & Assembler & IRCAM 4B \\
		SYN4B & 1978 & P . Prevot & DEC LSI­-11 & Assembler & IRCAM 4B \\
		& & N. Rolnick & & & \\
		4PLAY & 1978 & C. Abbott & DEC PDP­-11 & Pascal & IRCAM 4C \\
		Musbox & 1979 & G. Loy & DEC PDP­-10 & Sail & Samson Box \\
		& & W. Schottstaedt & & & \\
		4CED & 1980 & C. Abbott & DEC PDP­-11 & C & IRCAM 4C \\
		Music 1000 & 1980 & D. Wallraff & DEC LSI­-11 & Assembler & DMX­-1000 \\
		4X & 1981 & J. Kott & DEC PDP­-11 & Assembler & IRCAM 4X \\
		FMX & 1982 & C. Abbott & DEC VAX-11 & C & Lucasfilm ASP \\
		Music 400 & 1982 & M. Puckette & DEC PDP-11 & C & Analogic AP­-400 \\
		Cleo & 1983 & C. Abbott & Sun & C & Lucasfilm ASP \\
		Music 320 & 1983 & T. Hegg & MC68000 & Assembler & TI TMS 32010 \\
		Music 500 & 1984 & M. Puckette & DEC VAX­-11 & C & Analogic AP­-500 \\
		4XY & 1986 & R. Rowe & DEC VAX-11 & C & IRCAM 4X \\
		& & O. Koechlin & & & \\
		Csound & 1989 & N. Bailey & Inmos & Occam & Inmos \\
		& & A. Purvis & Transputer & & Transputer \\
		& & I. Bowler & & & \\
		& & P. Manning & & & \\
		Music 56000 & 1989 & K. Lent & IBM PS/2 & Assembler & Motorola \\
		& & R. Pinkston & & & DSP56001 \\
		& & P. Silsbee & & & \\
		NeXT Music and & 1989 & D. Jaffe & NeXT & Objective-C & Motorola \\
		Sound Kits & & L. Boynton & & & DSP56001 \\
		& & J. Smith & & & \\
		Digital Signal Patcher & 1990 & A. Pellecchia & IBM PC & C & TI TMS320C30 \\
		MUSIC30 & 1991 & J. Dashow & IBM PC & Prolog & TI TMS320C30 \\
		IRCAM Max & 1991 & M. Puckette & NeXT & C & Intel i860 \\
		IRIS Edit20 & 1992 & P . Andenacci & Atari SM1000 & C & MARS \\
		& & E. Favreau & & & \\
		& & N. Larosa & & & \\
		& & A. Prestigiacomo & & & \\
		& & C. Rosati & & & \\
		& & S. Sapir & & & \\
		Unison & 1992 & J. Bate & Apple & C & Motorola \\
		& & & Macintosh & C & DSP56001 \\
		\hline
	\end{tabular}
\end{table}

\begin{table}[htbp]
	\caption{Score input languages \cite[811]{Roads1995}.}
	\centering
	\vspace{12pt}
	\begin{tabular}{ *{3}{l} }
		\hline
		Language & Reference & Comments \\
		\hline
		DARMS & Erickson 1975 & Score analysis and archival \\
		MUSTRAN & Wenker 1972 & Score analysis and archival \\
		Standard Music & Newcomb and & Score analysis and archival \\
		Description Language & Goldfarb 1989 & \\
		SCORE & Smith 1972 & Score synthesis and printing \\
		Musica & De Poli 1978 & Score synthesis and printing \\
		SCRIPT & New England Digital 1981 & Score synthesis and printing \\
		Scriptu & Brown 1977 & Score synthesis and printing \\
		ASHTON & Ames 1985 & Score synthesis and printing \\
		Notepro & Beauchamp, Code and Chen 1990 & Score synthesis and printing \\
		Yet Another Music Input Language & Fry 1980 & Score synthesis and printing \\
		\hline
	\end{tabular}
\end{table}

%\begin{algorithm}
%\label{}
%\SetKw{KwThis}{this}
%\KwThis.length $\gets$ length\;
%\For{}{
%	\For{}{
%		\uIf{}{}
%		\uElseIf{}{}
%		\Else{}
%	}
%}
%\caption{}
%\end{algorithm}

\begin{table}[htbp]
	\caption{Procedural music composition languages \cite[815-817]{Roads1995}.}
	\centering
	\vspace{12pt}
	\begin{tabular}{ *{3}{l} }
		\hline
		Language & References & Comments \\
		\hline
		MUSICOMP & Baker 1963, Hiller and Leal 1966 & \\
		GROOVE & Mathews and Moore 1970 & \\
		TEMPO & Clough 1970 & \\
		SCORE & Smith 1972 & Preprocessor for \\
		&& Music V and MUS10 \\
		PLAY & Chadabe and Meyers 1978 & \\
		Tree and Cotree & Roads 1978b & \\
		GGDL & Holtzman 1981 & \\
		MPL & Nelson 1977, 1980 & Support for MIDI \\
		Pla & Schottstaedt 1983, 1989a & \\
		SAWDUST & Blum 1979, Hamlin and Roads 1985 & \\
		PILE & Berg 1979 & \\
		Flavors Band & Fry 1984 & \\
		FORMES & Rodet and Cointe 1984 & \\
		Formula & Andersen and Kuivila 1986, 1991 & Forth-based with \\
		&& real-time support \\
		MIDI­-LISP & Boynton et al. 1986 & Support for MIDI \\
		HMSL & Polansky, Rosenboom, and Burk 1987, & Forth-based with \\
		& Polansky et al. 1988 & MIDI support \\
		Personal Composer Lisp & Miller 1985 & Integrated MIDI sequencer \\
		Player & Loy 1986 & \\
		LOCO & Desain and Honing 1988 & Logo-based \\
		COMPOSE & Ames 1989b & \\
		Canon & Dannenberg 1989a & Lisp-based \\
		Hyperscore & Pope 1986a, b & Object-oriented \\
		MODE & Pope 1991a & Object-oriented \\
		Music Kit & Jaffe and Boynton 1989 & \\
		Lisp Kernel & Rahn 1990 & \\
		Keynote & Thompson 1990 & Support for MIDI \\
		Arctic & Dannenberg, McAvinney, and Rubine 1986 & Functional \\
		Esquisse & Baisnée et & Support for \\
		& al. 1988 & data structures \\
		Moxc & Dannenberg 1989b & \\
		Common Music & Taube 1991 & \\
		PatchWork & Laurson and Duthen 1989, Malt 1993, & Graphical patching \\
		& Barrière, Iovino, and Laurson 1991 & of Lisp functions \\
		Symbolic Composer & Tonality Systems 1993 & Lisp-based with \\
		&& MIDI output \\
		\hline
	\end{tabular}
\end{table}

%\chapter{}

\section{List of Terminal Symbols} \label{sec:symbols}
%\input{tex/appendixC}
%\input{tex/appendixD}

\bibliographystyle{bst/Science_Web}
%\bibliographystyle{bst/apa-good}
%\bibliographystyle{bst/Chicago_Web}
%\bibliographystyle{bst/ecology_web}
%\bibliographystyle{bst/IEEEtran}
%\bibliographystyle{bst/mla-good}

\bibliography{references}

\biography{
	Bio goes here.
}

\end{document}
